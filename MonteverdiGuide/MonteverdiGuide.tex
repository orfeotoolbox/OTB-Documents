%%%%%%%%%%%%%%%%%%%%%%%%%%%%%%%%%%%%%%%%%%%%%%%%%%%%%%%%%%%%%%%%%%%%%
%
% Complete documentation on the extended LaTeX markup used for Insight
% documentation is available in ``Documenting Insight'', that is part
% of the standard documentation for Insight.  It may be found online
% at:
%
%                    http://www.itk.org
%
%%%%%%%%%%%%%%%%%%%%%%%%%%%%%%%%%%%%%%%%%%%%%%%%%%%%%%%%%%%%%%%%%%%%%

\documentclass{InsightSoftwareGuide}


\usepackage[dvips]{graphicx}
\usepackage{times,lscape,url}
%\usepackage{mdwtab}


%%% \usepackage[latin1]{inputenc}
%%% \selectlanguage{french}
% Configuration pour les accents francais pour l'OTB
\usepackage[latin1]{inputenc}
%\usepackage[french]{babel}
\usepackage{tikz}


\usepackage{color}

\definecolor{listcomment}{rgb}{0.0,0.5,0.0}
\definecolor{listkeyword}{rgb}{0.0,0.0,0.5}
\definecolor{listnumbers}{gray}{0.65}
\definecolor{listlightgray}{gray}{0.955}
\definecolor{listwhite}{gray}{1.0}

\usepackage{listings}
\newcommand{\lstsetcpp}
{
\lstset{frame = tb,
       framerule = 0.25pt,
       float,
       fontadjust,
       backgroundcolor={\color{listlightgray}},
       basicstyle = {\ttfamily\footnotesize},
       keywordstyle = {\ttfamily\color{listkeyword}\textbf},
       identifierstyle = {\ttfamily},
       commentstyle = {\ttfamily\color{listcomment}\textit},
       stringstyle = {\ttfamily},
       showstringspaces = false,
       showtabs = false,
       numbers = none,
       numbersep = 6pt,
       numberstyle={\ttfamily\color{listnumbers}},
       tabsize = 2,
       language=[ANSI]C++,
       floatplacement=!h
       }
}
\newcommand{\lstsetpython}
{
\lstset{language=Python
        }
}
\newcommand{\lstsetjava}
{
\lstset{language=Java
        }
}


\newif\ifitkFullVersion
\itkFullVersiontrue
%\itkFullVersionfalse

\newif\ifitkPrintedVersion
\itkPrintedVersiontrue
%\itkPrintedVersionfalse


%%%%%%%%%%%%%%%%%%%%%%%%%%%%%%%%%%%%%%%%%%%%%%%%%%%%%%%%%%%%%%%%%%
%
%  hyperref should be the last package to be loaded.
%
%%%%%%%%%%%%%%%%%%%%%%%%%%%%%%%%%%%%%%%%%%%%%%%%%%%%%%%%%%%%%%%%%%
\ifitkPrintedVersion
\usepackage[dvips,
pdftitle={Monteverdi User Guide},
pdfauthor={CNES},
pdfsubject={Remote Sensing, Orfeo, Pleiades, Cosmo Skymed},
pdfkeywords={mage processing, Remote sensing, Guide},
pdfpagemode={UseOutlines},
bookmarks,bookmarksopen,
pdfstartview={FitH},
backref,
colorlinks,linkcolor={black},citecolor={black},urlcolor={black},
]{hyperref}
\else
\usepackage[dvips,
pdftitle={Monteverdi User Guide},
pdfauthor={CNES},
pdfsubject={Remote Sensing, Orfeo, Pleiades, Cosmo Skymed},
pdfkeywords={mage processing, Remote sensing, Guide},
pdfpagemode={UseOutlines},
bookmarks,bookmarksopen,
pdfstartview={FitH},
backref,
colorlinks,linkcolor={blue},citecolor={blue},urlcolor={blue},
]{hyperref}
\fi

\usepackage{amsmath,amssymb,amsfonts}
\usepackage{bbm}
%%%%%%%%%%%%%%%%%%%%%%%%%%%%%%%%%%%%%%%%%%%%%%%%%%%%%%%%%%%%%%%%%%%
%
%
%   Load configuration parameters prepared by CMake
%
%
%%%%%%%%%%%%%%%%%%%%%%%%%%%%%%%%%%%%%%%%%%%%%%%%%%%%%%%%%%%%%%%%%%%

\input{../SoftwareGuide/SoftwareGuideConfiguration.tex.in}

%\def\logoCNES{../Art/CNES_nom.eps}

\newtheorem{algo}{Algorithm}
\newtheorem{defin}{Definition}
%%%%%%%%%%%%%%%%%%%%%%%%%%%%%%%%%%%%%%%%%%%%%%%%%%%%%%%%%%%%%%%%%%%
%
%
%           The Insight Toolkit Software Guide
%
%
%%%%%%%%%%%%%%%%%%%%%%%%%%%%%%%%%%%%%%%%%%%%%%%%%%%%%%%%%%%%%%%%%%%

\title{The Monteverdi User Guide\\ Updated
  for OTB-3.2}

\author{OTB Development Team}

\authoraddress{
  \url{http://www.orfeo-toolbox.org/Monteverdi}\\
  e-mail: \email{otb@cnes.fr}
}

\date{\today}


% actually write the .idx file
\makeindex

\setcounter{tocdepth}{3}



%%%%%%%%%%%%%%%%%%%%%%%%%%%%%%%%%%%%%%%%%%%%%%%%%%%%%%%%%%%%%%%%%%%
%
%           Begin Document
%
%%%%%%%%%%%%%%%%%%%%%%%%%%%%%%%%%%%%%%%%%%%%%%%%%%%%%%%%%%%%%%%%%%%


\begin{document}

\title{Monteverdi Guide}
\author{OTB Development Team}


%\maketitle

\tableofcontents

%%%%%%%%%%%%%%%%%%%%%%%%%%%%%%%%%%%%%%%%%%%%%%%%%%%%%%%%%%%%%%%%%%%%%%
%
% Complete documentation on the extended LaTeX markup used for Insight
% documentation is available in ``Documenting Insight'', that is part
% of the standard documentation for Insight.  It may be found online
% at:
%
%                    http://www.itk.org
%
%%%%%%%%%%%%%%%%%%%%%%%%%%%%%%%%%%%%%%%%%%%%%%%%%%%%%%%%%%%%%%%%%%%%%

\documentclass{InsightSoftwareGuide}


\usepackage[dvips]{graphicx}
\usepackage{times,lscape,url}
%\usepackage{mdwtab}


%%% \usepackage[latin1]{inputenc}
%%% \selectlanguage{french}
% Configuration pour les accents francais pour l'OTB
\usepackage[latin1]{inputenc}
%\usepackage[french]{babel}
\usepackage{tikz}


\usepackage{color}

\definecolor{listcomment}{rgb}{0.0,0.5,0.0}
\definecolor{listkeyword}{rgb}{0.0,0.0,0.5}
\definecolor{listnumbers}{gray}{0.65}
\definecolor{listlightgray}{gray}{0.955}
\definecolor{listwhite}{gray}{1.0}

\usepackage{listings}
\newcommand{\lstsetcpp}
{
\lstset{frame = tb,
       framerule = 0.25pt,
       float,
       fontadjust,
       backgroundcolor={\color{listlightgray}},
       basicstyle = {\ttfamily\footnotesize},
       keywordstyle = {\ttfamily\color{listkeyword}\textbf},
       identifierstyle = {\ttfamily},
       commentstyle = {\ttfamily\color{listcomment}\textit},
       stringstyle = {\ttfamily},
       showstringspaces = false,
       showtabs = false,
       numbers = none,
       numbersep = 6pt,
       numberstyle={\ttfamily\color{listnumbers}},
       tabsize = 2,
       language=[ANSI]C++,
       floatplacement=!h
       }
}
\newcommand{\lstsetpython}
{
\lstset{language=Python
        }
}
\newcommand{\lstsetjava}
{
\lstset{language=Java
        }
}


\newif\ifitkFullVersion
\itkFullVersiontrue
%\itkFullVersionfalse

\newif\ifitkPrintedVersion
\itkPrintedVersiontrue
%\itkPrintedVersionfalse


%%%%%%%%%%%%%%%%%%%%%%%%%%%%%%%%%%%%%%%%%%%%%%%%%%%%%%%%%%%%%%%%%%
%
%  hyperref should be the last package to be loaded.
%
%%%%%%%%%%%%%%%%%%%%%%%%%%%%%%%%%%%%%%%%%%%%%%%%%%%%%%%%%%%%%%%%%%
\ifitkPrintedVersion
\usepackage[dvips,
pdftitle={Monteverdi User Guide},
pdfauthor={CNES},
pdfsubject={Remote Sensing, Orfeo, Pleiades, Cosmo Skymed},
pdfkeywords={mage processing, Remote sensing, Guide},
pdfpagemode={UseOutlines},
bookmarks,bookmarksopen,
pdfstartview={FitH},
backref,
colorlinks,linkcolor={black},citecolor={black},urlcolor={black},
]{hyperref}
\else
\usepackage[dvips,
pdftitle={Monteverdi User Guide},
pdfauthor={CNES},
pdfsubject={Remote Sensing, Orfeo, Pleiades, Cosmo Skymed},
pdfkeywords={mage processing, Remote sensing, Guide},
pdfpagemode={UseOutlines},
bookmarks,bookmarksopen,
pdfstartview={FitH},
backref,
colorlinks,linkcolor={blue},citecolor={blue},urlcolor={blue},
]{hyperref}
\fi

\usepackage{amsmath,amssymb,amsfonts}
\usepackage{bbm}
%%%%%%%%%%%%%%%%%%%%%%%%%%%%%%%%%%%%%%%%%%%%%%%%%%%%%%%%%%%%%%%%%%%
%
%
%   Load configuration parameters prepared by CMake
%
%
%%%%%%%%%%%%%%%%%%%%%%%%%%%%%%%%%%%%%%%%%%%%%%%%%%%%%%%%%%%%%%%%%%%

\input{../SoftwareGuide/SoftwareGuideConfiguration.tex.in}

%\def\logoCNES{../Art/CNES_nom.eps}

\newtheorem{algo}{Algorithm}
\newtheorem{defin}{Definition}
%%%%%%%%%%%%%%%%%%%%%%%%%%%%%%%%%%%%%%%%%%%%%%%%%%%%%%%%%%%%%%%%%%%
%
%
%           The Insight Toolkit Software Guide
%
%
%%%%%%%%%%%%%%%%%%%%%%%%%%%%%%%%%%%%%%%%%%%%%%%%%%%%%%%%%%%%%%%%%%%

\title{The Monteverdi User Guide\\ Updated
  for OTB-3.2}

\author{OTB Development Team}

\authoraddress{
  \url{http://www.orfeo-toolbox.org/Monteverdi}\\
  e-mail: \email{otb@cnes.fr}
}

\date{\today}


% actually write the .idx file
\makeindex

\setcounter{tocdepth}{3}



%%%%%%%%%%%%%%%%%%%%%%%%%%%%%%%%%%%%%%%%%%%%%%%%%%%%%%%%%%%%%%%%%%%
%
%           Begin Document
%
%%%%%%%%%%%%%%%%%%%%%%%%%%%%%%%%%%%%%%%%%%%%%%%%%%%%%%%%%%%%%%%%%%%


\begin{document}

\title{Monteverdi Guide}
\author{OTB Development Team}


%\maketitle

\tableofcontents

%%%%%%%%%%%%%%%%%%%%%%%%%%%%%%%%%%%%%%%%%%%%%%%%%%%%%%%%%%%%%%%%%%%%%%
%
% Complete documentation on the extended LaTeX markup used for Insight
% documentation is available in ``Documenting Insight'', that is part
% of the standard documentation for Insight.  It may be found online
% at:
%
%                    http://www.itk.org
%
%%%%%%%%%%%%%%%%%%%%%%%%%%%%%%%%%%%%%%%%%%%%%%%%%%%%%%%%%%%%%%%%%%%%%

\documentclass{InsightSoftwareGuide}


\usepackage[dvips]{graphicx}
\usepackage{times,lscape,url}
%\usepackage{mdwtab}


%%% \usepackage[latin1]{inputenc}
%%% \selectlanguage{french}
% Configuration pour les accents francais pour l'OTB
\usepackage[latin1]{inputenc}
%\usepackage[french]{babel}
\usepackage{tikz}


\usepackage{color}

\definecolor{listcomment}{rgb}{0.0,0.5,0.0}
\definecolor{listkeyword}{rgb}{0.0,0.0,0.5}
\definecolor{listnumbers}{gray}{0.65}
\definecolor{listlightgray}{gray}{0.955}
\definecolor{listwhite}{gray}{1.0}

\usepackage{listings}
\newcommand{\lstsetcpp}
{
\lstset{frame = tb,
       framerule = 0.25pt,
       float,
       fontadjust,
       backgroundcolor={\color{listlightgray}},
       basicstyle = {\ttfamily\footnotesize},
       keywordstyle = {\ttfamily\color{listkeyword}\textbf},
       identifierstyle = {\ttfamily},
       commentstyle = {\ttfamily\color{listcomment}\textit},
       stringstyle = {\ttfamily},
       showstringspaces = false,
       showtabs = false,
       numbers = none,
       numbersep = 6pt,
       numberstyle={\ttfamily\color{listnumbers}},
       tabsize = 2,
       language=[ANSI]C++,
       floatplacement=!h
       }
}
\newcommand{\lstsetpython}
{
\lstset{language=Python
        }
}
\newcommand{\lstsetjava}
{
\lstset{language=Java
        }
}


\newif\ifitkFullVersion
\itkFullVersiontrue
%\itkFullVersionfalse

\newif\ifitkPrintedVersion
\itkPrintedVersiontrue
%\itkPrintedVersionfalse


%%%%%%%%%%%%%%%%%%%%%%%%%%%%%%%%%%%%%%%%%%%%%%%%%%%%%%%%%%%%%%%%%%
%
%  hyperref should be the last package to be loaded.
%
%%%%%%%%%%%%%%%%%%%%%%%%%%%%%%%%%%%%%%%%%%%%%%%%%%%%%%%%%%%%%%%%%%
\ifitkPrintedVersion
\usepackage[dvips,
pdftitle={Monteverdi User Guide},
pdfauthor={CNES},
pdfsubject={Remote Sensing, Orfeo, Pleiades, Cosmo Skymed},
pdfkeywords={mage processing, Remote sensing, Guide},
pdfpagemode={UseOutlines},
bookmarks,bookmarksopen,
pdfstartview={FitH},
backref,
colorlinks,linkcolor={black},citecolor={black},urlcolor={black},
]{hyperref}
\else
\usepackage[dvips,
pdftitle={Monteverdi User Guide},
pdfauthor={CNES},
pdfsubject={Remote Sensing, Orfeo, Pleiades, Cosmo Skymed},
pdfkeywords={mage processing, Remote sensing, Guide},
pdfpagemode={UseOutlines},
bookmarks,bookmarksopen,
pdfstartview={FitH},
backref,
colorlinks,linkcolor={blue},citecolor={blue},urlcolor={blue},
]{hyperref}
\fi

\usepackage{amsmath,amssymb,amsfonts}
\usepackage{bbm}
%%%%%%%%%%%%%%%%%%%%%%%%%%%%%%%%%%%%%%%%%%%%%%%%%%%%%%%%%%%%%%%%%%%
%
%
%   Load configuration parameters prepared by CMake
%
%
%%%%%%%%%%%%%%%%%%%%%%%%%%%%%%%%%%%%%%%%%%%%%%%%%%%%%%%%%%%%%%%%%%%

\input{../SoftwareGuide/SoftwareGuideConfiguration.tex.in}

%\def\logoCNES{../Art/CNES_nom.eps}

\newtheorem{algo}{Algorithm}
\newtheorem{defin}{Definition}
%%%%%%%%%%%%%%%%%%%%%%%%%%%%%%%%%%%%%%%%%%%%%%%%%%%%%%%%%%%%%%%%%%%
%
%
%           The Insight Toolkit Software Guide
%
%
%%%%%%%%%%%%%%%%%%%%%%%%%%%%%%%%%%%%%%%%%%%%%%%%%%%%%%%%%%%%%%%%%%%

\title{The Monteverdi User Guide\\ Updated
  for OTB-3.2}

\author{OTB Development Team}

\authoraddress{
  \url{http://www.orfeo-toolbox.org/Monteverdi}\\
  e-mail: \email{otb@cnes.fr}
}

\date{\today}


% actually write the .idx file
\makeindex

\setcounter{tocdepth}{3}



%%%%%%%%%%%%%%%%%%%%%%%%%%%%%%%%%%%%%%%%%%%%%%%%%%%%%%%%%%%%%%%%%%%
%
%           Begin Document
%
%%%%%%%%%%%%%%%%%%%%%%%%%%%%%%%%%%%%%%%%%%%%%%%%%%%%%%%%%%%%%%%%%%%


\begin{document}

\title{Monteverdi Guide}
\author{OTB Development Team}


%\maketitle

\tableofcontents

%%%%%%%%%%%%%%%%%%%%%%%%%%%%%%%%%%%%%%%%%%%%%%%%%%%%%%%%%%%%%%%%%%%%%%
%
% Complete documentation on the extended LaTeX markup used for Insight
% documentation is available in ``Documenting Insight'', that is part
% of the standard documentation for Insight.  It may be found online
% at:
%
%                    http://www.itk.org
%
%%%%%%%%%%%%%%%%%%%%%%%%%%%%%%%%%%%%%%%%%%%%%%%%%%%%%%%%%%%%%%%%%%%%%

\documentclass{InsightSoftwareGuide}


\usepackage[dvips]{graphicx}
\usepackage{times,lscape,url}
%\usepackage{mdwtab}


%%% \usepackage[latin1]{inputenc}
%%% \selectlanguage{french}
% Configuration pour les accents francais pour l'OTB
\usepackage[latin1]{inputenc}
%\usepackage[french]{babel}
\usepackage{tikz}


\usepackage{color}

\definecolor{listcomment}{rgb}{0.0,0.5,0.0}
\definecolor{listkeyword}{rgb}{0.0,0.0,0.5}
\definecolor{listnumbers}{gray}{0.65}
\definecolor{listlightgray}{gray}{0.955}
\definecolor{listwhite}{gray}{1.0}

\usepackage{listings}
\newcommand{\lstsetcpp}
{
\lstset{frame = tb,
       framerule = 0.25pt,
       float,
       fontadjust,
       backgroundcolor={\color{listlightgray}},
       basicstyle = {\ttfamily\footnotesize},
       keywordstyle = {\ttfamily\color{listkeyword}\textbf},
       identifierstyle = {\ttfamily},
       commentstyle = {\ttfamily\color{listcomment}\textit},
       stringstyle = {\ttfamily},
       showstringspaces = false,
       showtabs = false,
       numbers = none,
       numbersep = 6pt,
       numberstyle={\ttfamily\color{listnumbers}},
       tabsize = 2,
       language=[ANSI]C++,
       floatplacement=!h
       }
}
\newcommand{\lstsetpython}
{
\lstset{language=Python
        }
}
\newcommand{\lstsetjava}
{
\lstset{language=Java
        }
}


\newif\ifitkFullVersion
\itkFullVersiontrue
%\itkFullVersionfalse

\newif\ifitkPrintedVersion
\itkPrintedVersiontrue
%\itkPrintedVersionfalse


%%%%%%%%%%%%%%%%%%%%%%%%%%%%%%%%%%%%%%%%%%%%%%%%%%%%%%%%%%%%%%%%%%
%
%  hyperref should be the last package to be loaded.
%
%%%%%%%%%%%%%%%%%%%%%%%%%%%%%%%%%%%%%%%%%%%%%%%%%%%%%%%%%%%%%%%%%%
\ifitkPrintedVersion
\usepackage[dvips,
pdftitle={Monteverdi User Guide},
pdfauthor={CNES},
pdfsubject={Remote Sensing, Orfeo, Pleiades, Cosmo Skymed},
pdfkeywords={mage processing, Remote sensing, Guide},
pdfpagemode={UseOutlines},
bookmarks,bookmarksopen,
pdfstartview={FitH},
backref,
colorlinks,linkcolor={black},citecolor={black},urlcolor={black},
]{hyperref}
\else
\usepackage[dvips,
pdftitle={Monteverdi User Guide},
pdfauthor={CNES},
pdfsubject={Remote Sensing, Orfeo, Pleiades, Cosmo Skymed},
pdfkeywords={mage processing, Remote sensing, Guide},
pdfpagemode={UseOutlines},
bookmarks,bookmarksopen,
pdfstartview={FitH},
backref,
colorlinks,linkcolor={blue},citecolor={blue},urlcolor={blue},
]{hyperref}
\fi

\usepackage{amsmath,amssymb,amsfonts}
\usepackage{bbm}
%%%%%%%%%%%%%%%%%%%%%%%%%%%%%%%%%%%%%%%%%%%%%%%%%%%%%%%%%%%%%%%%%%%
%
%
%   Load configuration parameters prepared by CMake
%
%
%%%%%%%%%%%%%%%%%%%%%%%%%%%%%%%%%%%%%%%%%%%%%%%%%%%%%%%%%%%%%%%%%%%

\input{../SoftwareGuide/SoftwareGuideConfiguration.tex.in}

%\def\logoCNES{../Art/CNES_nom.eps}

\newtheorem{algo}{Algorithm}
\newtheorem{defin}{Definition}
%%%%%%%%%%%%%%%%%%%%%%%%%%%%%%%%%%%%%%%%%%%%%%%%%%%%%%%%%%%%%%%%%%%
%
%
%           The Insight Toolkit Software Guide
%
%
%%%%%%%%%%%%%%%%%%%%%%%%%%%%%%%%%%%%%%%%%%%%%%%%%%%%%%%%%%%%%%%%%%%

\title{The Monteverdi User Guide\\ Updated
  for OTB-3.2}

\author{OTB Development Team}

\authoraddress{
  \url{http://www.orfeo-toolbox.org/Monteverdi}\\
  e-mail: \email{otb@cnes.fr}
}

\date{\today}


% actually write the .idx file
\makeindex

\setcounter{tocdepth}{3}



%%%%%%%%%%%%%%%%%%%%%%%%%%%%%%%%%%%%%%%%%%%%%%%%%%%%%%%%%%%%%%%%%%%
%
%           Begin Document
%
%%%%%%%%%%%%%%%%%%%%%%%%%%%%%%%%%%%%%%%%%%%%%%%%%%%%%%%%%%%%%%%%%%%


\begin{document}

\title{Monteverdi Guide}
\author{OTB Development Team}


%\maketitle

\tableofcontents

%\input{../SoftwareGuide/Latex/MonteverdiGuide.tex}

\section{Introduction to Monteverdi}
\section{Getting started}
\subsection{Why this software is called Monteverdi?}
The application is called Monteverdi, since this is the name of the Orfeo composer.The application allows you to build interactivelly remote sensing processes based on the Orfeo Toolbox library. This is also in remebering of the great (and once open source) Khoros/Cantata software.
\subsection{Installation}
Installation of Monteverdi is very simple. Standard installer packages are available for now only on MS Windows. For many flavors of GNU/Linux binary packages (rpm and deb) or software repositories
to add to your installation manager will be provided soon. Get the latest information on binary packages at
the OTB website at http://www.orfeo-toolbox.org/otb/download.html.

\subsection{Installation from source}
If you need to build Monteverdi from source, please refer to the coding and compiling guide available at
http://www.orfeo-toolbox.org/otb/documentation.html.
 
\section{Anatomy of the application}
\subsection{What does it look like}
This is Monteverdi's main window where the menus are available and where you can see the different modules which have been set up for the processing. Input data are obtained by readers. When you choose to use a new module, you select its input data, and therefore, you build a processing pipeline sequentially. Let's have a look at the different menus. The first one is of course the "File" menu. This menu allows you to open a data set, to save it and to cache it. The "data set" concept is interesting, since you don't need to define by hand if you are looking for an image or a vector file. Of course, you don't need to do anything special for any particular file format. So opening a data set will create a "reader" which will appear in the main window. At any time, you can use the "save data set" option in order to store to a file the result of any processing module.
\subsection{Open an image with Monteverdi}
\subsection{Visualize an image with Monteverdi}
\subsection{Cache dataset}
The "cache data set" is a very interesting thing. As you know, OTB implements processing on demand, so when you build a processing pipeline, no processing takes place unless you ask for it explicitly. That means that you can plug together the opening of a data set, an orthorectification and a spleckle filter, for example, but nothing will really be computed until you trigger the pipeline execution. This is very convenient, since you can quicly build a processing pipeline and let it execute afterwards while you have a coffee. In Monteverdi, you execute the processing by saving the result of the last module of a pipeline. However, sometimes, you may want to execute a part of the pipeline without wanting to give a name to the obtained result. You can do this by caching a data set. That is, the result will be stored in a temporary file which will be created in the "Caching" directory created by the application. Another situation in which you may need to cache a data set is when you need the input of a module to exist when you set its parameters. This is nor a real requirement, since Monteverdi will generate the needed data by streaming it, but this can be inefficient. This for instance about visualization of the result of a complex processing. Using streaming for brwsing through the result image means processing the visible part every time you move inside the image. Caching the data before visualization generated the whole data set in advance allowing for a more swift display. All modules allow you to cache their input data sets.
\section{Available modules}
\subsection{I/O operations}
\subsubsection{Extract region of interest}
extracting regions of interest (ROI) from an image.
\subsubsection{Concatenate image bands}
ne for concatenating images into one single multi-band image (they need to have the same size).

The pink button to the right of the image selection menu indicates that the image has not been generated (streamed). If you push the button, the image will be cached. 
\subsection{Image Vizualization}
\subsection{Geometric process}

\subsection{Calibration}
\subsection{Filtering Operations}
\subsection{Learning}
\subsection{Specific SAR functionnalities}


\section{Developper Guide}
\subsection{Build your custom Monteverdi module}
\end{document}

\section{Introduction to Monteverdi}
\section{Getting started}
\subsection{Why this software is called Monteverdi?}
The application is called Monteverdi, since this is the name of the Orfeo composer.The application allows you to build interactivelly remote sensing processes based on the Orfeo Toolbox library. This is also in remebering of the great (and once open source) Khoros/Cantata software.
\subsection{Installation}
Installation of Monteverdi is very simple. Standard installer packages are available for now only on MS Windows. For many flavors of GNU/Linux binary packages (rpm and deb) or software repositories
to add to your installation manager will be provided soon. Get the latest information on binary packages at
the OTB website at http://www.orfeo-toolbox.org/otb/download.html.

\subsection{Installation from source}
If you need to build Monteverdi from source, please refer to the coding and compiling guide available at
http://www.orfeo-toolbox.org/otb/documentation.html.
 
\section{Anatomy of the application}
\subsection{What does it look like}
This is Monteverdi's main window where the menus are available and where you can see the different modules which have been set up for the processing. Input data are obtained by readers. When you choose to use a new module, you select its input data, and therefore, you build a processing pipeline sequentially. Let's have a look at the different menus. The first one is of course the "File" menu. This menu allows you to open a data set, to save it and to cache it. The "data set" concept is interesting, since you don't need to define by hand if you are looking for an image or a vector file. Of course, you don't need to do anything special for any particular file format. So opening a data set will create a "reader" which will appear in the main window. At any time, you can use the "save data set" option in order to store to a file the result of any processing module.
\subsection{Open an image with Monteverdi}
\subsection{Visualize an image with Monteverdi}
\subsection{Cache dataset}
The "cache data set" is a very interesting thing. As you know, OTB implements processing on demand, so when you build a processing pipeline, no processing takes place unless you ask for it explicitly. That means that you can plug together the opening of a data set, an orthorectification and a spleckle filter, for example, but nothing will really be computed until you trigger the pipeline execution. This is very convenient, since you can quicly build a processing pipeline and let it execute afterwards while you have a coffee. In Monteverdi, you execute the processing by saving the result of the last module of a pipeline. However, sometimes, you may want to execute a part of the pipeline without wanting to give a name to the obtained result. You can do this by caching a data set. That is, the result will be stored in a temporary file which will be created in the "Caching" directory created by the application. Another situation in which you may need to cache a data set is when you need the input of a module to exist when you set its parameters. This is nor a real requirement, since Monteverdi will generate the needed data by streaming it, but this can be inefficient. This for instance about visualization of the result of a complex processing. Using streaming for brwsing through the result image means processing the visible part every time you move inside the image. Caching the data before visualization generated the whole data set in advance allowing for a more swift display. All modules allow you to cache their input data sets.
\section{Available modules}
\subsection{I/O operations}
\subsubsection{Extract region of interest}
extracting regions of interest (ROI) from an image.
\subsubsection{Concatenate image bands}
ne for concatenating images into one single multi-band image (they need to have the same size).

The pink button to the right of the image selection menu indicates that the image has not been generated (streamed). If you push the button, the image will be cached. 
\subsection{Image Vizualization}
\subsection{Geometric process}

\subsection{Calibration}
\subsection{Filtering Operations}
\subsection{Learning}
\subsection{Specific SAR functionnalities}


\section{Developper Guide}
\subsection{Build your custom Monteverdi module}
\end{document}

\section{Introduction to Monteverdi}
\section{Getting started}
\subsection{Why this software is called Monteverdi?}
The application is called Monteverdi, since this is the name of the Orfeo composer.The application allows you to build interactivelly remote sensing processes based on the Orfeo Toolbox library. This is also in remebering of the great (and once open source) Khoros/Cantata software.
\subsection{Installation}
Installation of Monteverdi is very simple. Standard installer packages are available for now only on MS Windows. For many flavors of GNU/Linux binary packages (rpm and deb) or software repositories
to add to your installation manager will be provided soon. Get the latest information on binary packages at
the OTB website at http://www.orfeo-toolbox.org/otb/download.html.

\subsection{Installation from source}
If you need to build Monteverdi from source, please refer to the coding and compiling guide available at
http://www.orfeo-toolbox.org/otb/documentation.html.
 
\section{Anatomy of the application}
\subsection{What does it look like}
This is Monteverdi's main window where the menus are available and where you can see the different modules which have been set up for the processing. Input data are obtained by readers. When you choose to use a new module, you select its input data, and therefore, you build a processing pipeline sequentially. Let's have a look at the different menus. The first one is of course the "File" menu. This menu allows you to open a data set, to save it and to cache it. The "data set" concept is interesting, since you don't need to define by hand if you are looking for an image or a vector file. Of course, you don't need to do anything special for any particular file format. So opening a data set will create a "reader" which will appear in the main window. At any time, you can use the "save data set" option in order to store to a file the result of any processing module.
\subsection{Open an image with Monteverdi}
\subsection{Visualize an image with Monteverdi}
\subsection{Cache dataset}
The "cache data set" is a very interesting thing. As you know, OTB implements processing on demand, so when you build a processing pipeline, no processing takes place unless you ask for it explicitly. That means that you can plug together the opening of a data set, an orthorectification and a spleckle filter, for example, but nothing will really be computed until you trigger the pipeline execution. This is very convenient, since you can quicly build a processing pipeline and let it execute afterwards while you have a coffee. In Monteverdi, you execute the processing by saving the result of the last module of a pipeline. However, sometimes, you may want to execute a part of the pipeline without wanting to give a name to the obtained result. You can do this by caching a data set. That is, the result will be stored in a temporary file which will be created in the "Caching" directory created by the application. Another situation in which you may need to cache a data set is when you need the input of a module to exist when you set its parameters. This is nor a real requirement, since Monteverdi will generate the needed data by streaming it, but this can be inefficient. This for instance about visualization of the result of a complex processing. Using streaming for brwsing through the result image means processing the visible part every time you move inside the image. Caching the data before visualization generated the whole data set in advance allowing for a more swift display. All modules allow you to cache their input data sets.
\section{Available modules}
\subsection{I/O operations}
\subsubsection{Extract region of interest}
extracting regions of interest (ROI) from an image.
\subsubsection{Concatenate image bands}
ne for concatenating images into one single multi-band image (they need to have the same size).

The pink button to the right of the image selection menu indicates that the image has not been generated (streamed). If you push the button, the image will be cached. 
\subsection{Image Vizualization}
\subsection{Geometric process}

\subsection{Calibration}
\subsection{Filtering Operations}
\subsection{Learning}
\subsection{Specific SAR functionnalities}


\section{Developper Guide}
\subsection{Build your custom Monteverdi module}
\end{document}

\section{Introduction to Monteverdi}
\section{Getting started}
\subsection{Why this software is called Monteverdi?}
The application is called Monteverdi, since this is the name of the Orfeo composer.The application allows you to build interactivelly remote sensing processes based on the Orfeo Toolbox library. This is also in remebering of the great (and once open source) Khoros/Cantata software.
\subsection{Installation}
Installation of Monteverdi is very simple. Standard installer packages are available for now only on MS Windows. For many flavors of GNU/Linux binary packages (rpm and deb) or software repositories
to add to your installation manager will be provided soon. Get the latest information on binary packages at
the OTB website at http://www.orfeo-toolbox.org/otb/download.html.

\subsection{Installation from source}
If you need to build Monteverdi from source, please refer to the coding and compiling guide available at
http://www.orfeo-toolbox.org/otb/documentation.html.
 
\section{Anatomy of the application}
\subsection{What does it look like}
This is Monteverdi's main window where the menus are available and where you can see the different modules which have been set up for the processing. Input data are obtained by readers. When you choose to use a new module, you select its input data, and therefore, you build a processing pipeline sequentially. Let's have a look at the different menus. The first one is of course the "File" menu. This menu allows you to open a data set, to save it and to cache it. The "data set" concept is interesting, since you don't need to define by hand if you are looking for an image or a vector file. Of course, you don't need to do anything special for any particular file format. So opening a data set will create a "reader" which will appear in the main window. At any time, you can use the "save data set" option in order to store to a file the result of any processing module.
\subsection{Open an image with Monteverdi}
\subsection{Visualize an image with Monteverdi}
\subsection{Cache dataset}
The "cache data set" is a very interesting thing. As you know, OTB implements processing on demand, so when you build a processing pipeline, no processing takes place unless you ask for it explicitly. That means that you can plug together the opening of a data set, an orthorectification and a spleckle filter, for example, but nothing will really be computed until you trigger the pipeline execution. This is very convenient, since you can quicly build a processing pipeline and let it execute afterwards while you have a coffee. In Monteverdi, you execute the processing by saving the result of the last module of a pipeline. However, sometimes, you may want to execute a part of the pipeline without wanting to give a name to the obtained result. You can do this by caching a data set. That is, the result will be stored in a temporary file which will be created in the "Caching" directory created by the application. Another situation in which you may need to cache a data set is when you need the input of a module to exist when you set its parameters. This is nor a real requirement, since Monteverdi will generate the needed data by streaming it, but this can be inefficient. This for instance about visualization of the result of a complex processing. Using streaming for brwsing through the result image means processing the visible part every time you move inside the image. Caching the data before visualization generated the whole data set in advance allowing for a more swift display. All modules allow you to cache their input data sets.
\section{Available modules}
\subsection{I/O operations}
\subsubsection{Extract region of interest}
extracting regions of interest (ROI) from an image.
\subsubsection{Concatenate image bands}
ne for concatenating images into one single multi-band image (they need to have the same size).

The pink button to the right of the image selection menu indicates that the image has not been generated (streamed). If you push the button, the image will be cached. 
\subsection{Image Vizualization}
\subsection{Geometric process}

\subsection{Calibration}
\subsection{Filtering Operations}
\subsection{Learning}
\subsection{Specific SAR functionnalities}


\section{Developper Guide}
\subsection{Build your custom Monteverdi module}
\end{document}