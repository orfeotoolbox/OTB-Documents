\documentclass[compress]{beamer}
\mode<presentation>
{
 \usetheme{Vilanova}
}

\usepackage[french]{babel}

\usepackage[utf8]{inputenc}

\usepackage{times}
\usepackage[T1]{fontenc}

\usepackage{amsfonts}
\usepackage{amsmath}
\usepackage{amssymb}
\usepackage{tikz}
\usepackage{eurosym}
%\usepackage{url}
\usepackage[normal]{subfigure}
\newcommand{\goodgap}{%
	\hspace{\subfigtopskip}%
	\hspace{\subfigbottomskip}}



%\newtheorem{definition}{Definition}

\title{Traitement d'images de télédétection}
\subtitle{Extraction de primitives}




\author
{jordi.inglada@cesbio.cnes.fr}
\normalsize

\institute[Cesbio] % (optional, but mostly needed)
{\textsc{Centre d'Études Spatiales de la Biosphère, Toulouse, France}}

\date{}

\pgfdeclareimage[height=96mm,width=128mm]{background}{fondsClairSansLogo}
\setbeamertemplate{background}{\pgfuseimage{background}}
\pgfdeclareimage[height=0.6cm]{logoIncrust}{logoIncrust}
\pgfdeclareimage[height=0.5cm]{logo_cesbio}{logo_cesbio}
\logo{
\begin{tabular}{lp{0.25\textwidth}lp{0.25\textwidth}r}
\href{http://www.cesbio.ups-tlse.fr/}{\pgfuseimage{logo_cesbio}}
&&\footnotesize{AUF - Marrakech 2011}&&
\href{http://www.orfeo-toolbox.org}{\pgfuseimage{logoIncrust}}\\
\end{tabular}
}


\subject{Extraction de primitives}




% Delete this, if you do not want the table of contents to pop up at
% the beginning of each subsection:
\AtBeginSubsection[]
{
  \begin{frame}<beamer>
    \frametitle{Outline}
    \tableofcontents[currentsection,currentsubsection]
  \end{frame}
}




% If you wish to uncover everything in a step-wise fashion, uncomment
% the following command: 

%\beamerdefaultoverlayspecification{<+->}
\begin{document}

\begin{frame}
  \titlepage
  \begin{center}
{\tiny Ce contenu est dérivé de la formation \href{http://www.orfeo-toolbox.org/packages/PragmaticRemoteSensing-handout.pdf}{``Pragmatic Remote
  Sensing''} dispensée par J. Inglada et E. Christophe en juillet 2010
  dans le cadre du colloque IGARSS. Il est mis à disposition selon les termes de la licence :\\
Creative Commons Paternité – Partage à l’Identique 3.0 non transcrit.} \href{http://creativecommons.org/licenses/by-sa/3.0/}{\includegraphics[width=0.05\textwidth]{/home/inglada/Dev/GH/IGARSS2010/Tutorial/Slides/Ressources/CC-licence.png}}    
  \end{center}
\end{frame}


\section[Intro]{La classification d'images}
\begin{frame}
\frametitle{La classification d'images}
  \begin{itemize}
  \item Définition : procédure par laquelle on attribue une étiquette
    aux objets (pixels de l'image)
  \item Supervisée
  \item Non-supervisée
  \item Orientée pixel
  \item Orientée objet
  \end{itemize}
\end{frame}

\begin{frame}
  \frametitle{Données pour la classification}
  \begin{itemize}
  \item Images (réflectances)
  \item Primitives
    \begin{itemize}
    \item Indices radiométriques: NDVI, brillance, couleur, angle
      spectral, etc.
    \item Statistiques, textures, etc.
    \item Transformations: ACP, MNF, ondelettes, etc.
    \end{itemize}
  \item Données exogènes
    \begin{itemize}
    \item MNT, cartes, etc.
    \end{itemize}
  \end{itemize}
\end{frame}

\begin{frame}
  \frametitle{La classification en 4 étapes}
  \begin{itemize}
  \item Sélection des attributs pertinents (primitives, etc.)
  \item Création d'un vecteur d'attributs par pixel
  \item Choix de l'étiquette de la classe (dans le cas supervisé)
  \item Apprentissage du classifieur
  \end{itemize}
\end{frame}
\section[Non-supervisé]{Classification non-supervisée}
\label{sec:unsupervised}
\begin{frame}
\frametitle{Classification non-supervisée}
  \begin{itemize}
  \item Aussi appelée {\em clustering}
  \item Nécessite une interprétation des résultats (reconnaissance des
    classes)
    \begin{itemize}
    \item Les étiquettes des classes sont des nombres (1, 2, ...)
    \end{itemize}
  \item Pas besoin de vérité terrain ou d'exemples
    \begin{itemize}
    \item Le nombre de classes est souvent choisi à la main
    \item Autres paramètres sont aussi nécessaires
    \end{itemize}
  \item Exemples: k-moyennes, ISO-Data, carte de Kohonen
  \end{itemize}
\end{frame}

\begin{frame}
  \frametitle{Exemple: K-moyenens}
\setbeamercovered{invisible}

{\tiny
  \begin{center}
    \begin{tabular}{lc}
     \onslide<2->{1. les k ``moyennes'' initiales sont} &\\
\onslide<2->{choisies aléatoirement.}&\\
&\onslide<2->{\includegraphics[width=0.15\textwidth]{kmeans_step1.pdf}}\\
     \onslide<3->{2. k clusters sont crées en associant chaque} &\\
     \onslide<3->{observation avec la moyenne la plus proche. Les
       partitions} & \\
     \onslide<3->{représentent le diagramme de Voronoï généré par les moyennes.} & \\
&     \onslide<3->{\includegraphics[width=0.15\textwidth]{K_Means_Example_Step_2.pdf}}\\
    \onslide<4->{3. Le centre de chaque classe devient la nouvelle moyenne.} & \\
&\onslide<4->{\includegraphics[width=0.15\textwidth]{K_Means_Example_Step_3.pdf}}\\
    \onslide<5->{Les étapes 2 et 3 sont répétées jusqu'à la convergence.} & \\
&\onslide<5->{\includegraphics[width=0.15\textwidth]{K_Means_Example_Step_4.pdf}}\\
    \end{tabular}
  \end{center}
{\tiny Images : Wikipedia}
}
\end{frame}

\begin{frame}
\frametitle{Exemple : K-moyennes à 5 classes }
\begin{columns}
\begin{column}{0.5\textwidth}
\begin{figure}[]
  \includegraphics[width=1.0\textwidth]{radio2-extract-3b.jpg}
\end{figure}
\end{column}
\begin{column}{0.5\textwidth}
\begin{figure}[]
  \includegraphics[width=1.0\textwidth]{kmeans-5-classes.png}
\end{figure}
\end{column}
\end{columns}
\end{frame}


\begin{frame}
  \frametitle{La main à la pâte}
  \begin{enumerate}
  \item Monteverdi: Learning $\rightarrow$ KMeans Clustering
  \item Choisir une image
  \item On peut utiliser seulement une fraction des pixels pour
    estimer les centroïdes
  \item Choisir le nombre de classes
  \item Fixer le nombre d'itérations et le seuil de convergence
  \end{enumerate}    
\end{frame}
\section[Supervisé]{Classification supervisée}
\label{sec:supervised}
\begin{frame}
  \frametitle{Classification supervisée}
  \begin{itemize}
  \item Nécessite des exemples ou une vérité terrain
  \item Les exemples peuvent avoir des étiquettes thématiques
    \begin{itemize}
    \item Différence entre occupation et utilisation des sols
    \end{itemize}
  \item Exemples: réseaux de neurones, maximum de vraisemblance, Support Vector Machines
  \end{itemize}
\end{frame}

\begin{frame}
  \frametitle{Exemple : SVM}
\setbeamercovered{invisible}
{\tiny
\begin{center}
  \begin{tabular}{cc}
\onslide<2->{H3 (vert) ne sépare pas les 2 classes.} & \onslide<5->{Hyperplan à marge maximale} \\
\onslide<3->{H1 (bleu) OK, mais petite marge.} &  \onslide<5->{SVM appris avec des échantillons de 2 classes.}\\
\onslide<4->{H2 (rouge) marge maximale.} & \onslide<5->{Échantillons
  dans la marge : vecteurs support.}\\
& \\
& \\
\includegraphics[width=0.3\textwidth]{Svm_separating_hyperplanes.png}
& \onslide<5->{\includegraphics[width=0.3\textwidth]{Svm_max_sep_hyperplane_with_margin.png}}\\
  \end{tabular}
\end{center}
{\tiny Images : Wikipedia}
}
\end{frame}

\begin{frame}
\frametitle{Exemple : SVM à 6 classes}
\framesubtitle{Eau, végétation, bâti, routes, nuages, ombres}
\begin{columns}
\begin{column}{0.5\textwidth}
\begin{figure}[]
  \includegraphics[width=1.0\textwidth]{radio2-extract-3b.jpg}
\end{figure}
\end{column}
\begin{column}{0.5\textwidth}
\begin{figure}[]
  \includegraphics[width=1.0\textwidth]{svm-6-classes.png}
\end{figure}
\end{column}
\end{columns}
\end{frame}


\begin{frame}
  \frametitle{La main à la pâte}
  \begin{enumerate}
  \item Monteverdi: Learning $\rightarrow$ SVM Classification
  \item Choisir l'image à classer
  \item Ajouter une classe
    \begin{itemize}
    \item On peut lui donner un nom et une couleur
    \end{itemize}
  \item Choisir des échantillons pour chaque classe
    \begin{itemize}
    \item Tracer des polygones et utiliser {\em End Polygon} pour les fermer
    \item On peut associer les polygones aux ensembles d'apprentissage
      et de test; ou choisir une sélection aléatoire
    \end{itemize}
  \item Learn
  \item Validate : affiche la matrice de confusion
  \item Display : affichage de l'image classée
  \end{enumerate}    
\end{frame}

\section[Orienté objet]{Classification orientée objet}
\label{sec:objectoriented}
\begin{frame}
  \frametitle{Classification orientée objet}
  \begin{itemize}
  \item Les pixels peuvent ne pas être appropriés pour décrire les
    classes d'intérêt
    \begin{itemize}
    \item la forme, la taille et d'autres caractéristiques des régions
      peuvent être plus pertinentes
    \end{itemize}
  \item Nous avons besoin de fournir au classifieur un ensemble de régions
    \begin{itemize}
    \item Segmentation d'images
    \end{itemize}
  \item Et leurs caractéristiques
    \begin{itemize}
    \item Calcul de primitives par région
    \end{itemize}
  \item Possibilité de faire de l'apprentissage actif
  \end{itemize}
\end{frame}

\begin{frame}
  \frametitle{Pas de main à la pâte}
Mais une démo si on a le temps!
\end{frame}


\end{document}
