\documentclass[compress]{beamer}
\mode<presentation>
{
 \usetheme{Vilanova}
}

\usepackage[french]{babel}

\usepackage[utf8]{inputenc}

\usepackage{times}
\usepackage[T1]{fontenc}

\usepackage{amsfonts}
\usepackage{amsmath}
\usepackage{amssymb}
\usepackage{tikz}
\usepackage{eurosym}
%\usepackage{url}
\usepackage[normal]{subfigure}
\newcommand{\goodgap}{%
	\hspace{\subfigtopskip}%
	\hspace{\subfigbottomskip}}



%\newtheorem{definition}{Definition}

\title{Traitement d'images de télédétection}
\subtitle{Détection de changements}




\author
{jordi.inglada@cesbio.cnes.fr}
\normalsize

\institute[Cesbio] % (optional, but mostly needed)
{\textsc{Centre d'Études Spatiales de la Biosphère, Toulouse, France}}

\date{}

\pgfdeclareimage[height=96mm,width=128mm]{background}{fondsClairSansLogo}
\setbeamertemplate{background}{\pgfuseimage{background}}
\pgfdeclareimage[height=0.6cm]{logoIncrust}{logoIncrust}
\pgfdeclareimage[height=0.5cm]{logo_cesbio}{logo_cesbio}
\logo{
\begin{tabular}{lp{0.25\textwidth}lp{0.25\textwidth}r}
\href{http://www.cesbio.ups-tlse.fr/}{\pgfuseimage{logo_cesbio}}
&&\footnotesize{AUF - Marrakech 2011}&&
\href{http://www.orfeo-toolbox.org}{\pgfuseimage{logoIncrust}}\\
\end{tabular}
}


\subject{Détection de changements}




% Delete this, if you do not want the table of contents to pop up at
% the beginning of each subsection:
\AtBeginSubsection[]
{
  \begin{frame}<beamer>
    \frametitle{Outline}
    \tableofcontents[currentsection,currentsubsection]
  \end{frame}
}




% If you wish to uncover everything in a step-wise fashion, uncomment
% the following command: 

%\beamerdefaultoverlayspecification{<+->}
\begin{document}

\begin{frame}
  \titlepage
  \begin{center}
{\tiny Ce contenu est dérivé de la formation \href{http://www.orfeo-toolbox.org/packages/PragmaticRemoteSensing-handout.pdf}{``Pragmatic Remote
  Sensing''} dispensée par J. Inglada et E. Christophe en juillet 2010
  dans le cadre du colloque IGARSS. Il est mis à disposition selon les termes de la licence :\\
Creative Commons Paternité – Partage à l’Identique 3.0 non transcrit.} \href{http://creativecommons.org/licenses/by-sa/3.0/}{\includegraphics[width=0.05\textwidth]{/home/inglada/Dev/GH/IGARSS2010/Tutorial/Slides/Ressources/CC-licence.png}}    
  \end{center}
\end{frame}


\section[Stratégies]{Stratégies classiques pour la détections de changements}

\begin{frame}
  \frametitle{Approches possibles}
\begin{itemize}

\item{Stratégie $1$ : détecteurs simples}

  Production d'une image de vraisemblance de changement (différences,
  ratios, etc.) et seuillage pour produire une carte binaire.

\item{Stratégie $2$ : Comparaison post-classification}

Génération de 2 cartes d'occupation des sols (une pour chaque date) et
comparaison des classes.

\item{Stratégie $3$ : Classification conjointe}

  Génération de la carte de changements directement à partir de la
  classification conjointe des 2 images.


\end{itemize}
\end{frame}


\section[Détecteurs]{Détecteurs disponibles dans l'OTB}
\begin{frame}
  \frametitle{Détecteurs disponibles}
  \small
  \begin{itemize}
    \item Différence pixel à pixel des valeurs moyennes dans les
      voisinages :
      \begin{equation}
	I_{D}(i,j)=I_{2}(i,j)-I_{1}(i,j).
      \end{equation}
    \item Ratio de moyennes locales :
      \begin{equation}
\displaystyle I_{R}(i,j) = 1 - min \left(\frac{\displaystyle I_{2}(i,j)}{\displaystyle I_{1}(i,j)},\frac{\displaystyle I_{1}(i,j)}{\displaystyle I_{2}(i,j)}\right).
\end{equation}
      \item Corrélation locale :
\begin{equation}
  I_\rho(i,j) = \frac{1}{N}\frac{\sum_{i,j}(I_1(i,j)-m_{I_1})(I_2(i,j)-m_{I_2})}{\sigma_{I_1}
\sigma_{I_2}}
\end{equation}
\item Distance de Kullback-Leibler entre les distributions locales
  \item Plusieurs versions de l'information mutuelle
  \end{itemize}
  \normalsize
\end{frame}

\begin{frame}
  \frametitle{La main à la pâte}
  \framesubtitle{Affichage des différences}
  \begin{enumerate}
  \item Monteverdi: File $\rightarrow$ Concatenate Images
  \item Choisir les amplitudes des 2 images et construire 1 image à 2 bandes
  \item Monteverdi: Visualization $\rightarrow$ Viewer
  \item Choisir l'image à 2 bandes
  \item Dans l'onglet {\em Setup}, choisir {\em RGB composition mode}
    et prendre 1,2,2.
  \item Interpréter les couleurs observées
  \item La même chose pourrait être faite en utilisant des images de primitives
  \end{enumerate}
\end{frame}

\begin{frame}
  \frametitle{La main à la pâte}
  \framesubtitle{Seuillage des différences}
  \begin{enumerate}
  \item Monteverdi: Filtering $\rightarrow$ Band Math
  \item Choisir les amplitudes des 2 images et calculer une différence
  \item Monteverdi: Filtering $\rightarrow$ Threshold
  \item Choisir l'image de différences et appliquer différents seuils
  \item La même chose pourrait être faite avec des ratios
  \item La même chose pourrait être faite en utilisant des images de primitives
  \end{enumerate}
\end{frame}


\section[Application]{Détection de changements interactive}

\begin{frame}
  \frametitle{Détection de changements interactive}
  \begin{itemize}
    \item Génération de cartes de changements binaires en utilisant une
      IHM
    \item Utilise des détecteurs simples en entrée
    \item L'opérateur donne des examples de {\em changement} et {\em non-changement}
    \item Classification supervisée par SVM
  \end{itemize}
\end{frame}


\begin{frame}
  \frametitle{Hands On}
  \framesubtitle{Joint Classification}
  \begin{enumerate}
  \item Monteverdi: Filtering $\rightarrow$ Change Detection
  \item Select the 2 images to process
    \begin{itemize}
    \item Raw images at 2 different dates
    \end{itemize}
  \item Uncheck {\em Use Change Detectors}
  \item Use the {\em Changed/Unchanged Class} buttons to select one of
    the classes
  \item Draw polygons on the images in order to give training samples
    to the algorithm
  \item {\em End Polygon} button is used to close the current polygon
  \item After selecting several polygons per class push {\em Learn}
  \item The {\em Display Results} button will show the result change detection
  \end{enumerate}
\end{frame}

\begin{frame}
  \frametitle{Hands On}
  \framesubtitle{Joint Classification with Change Detectors}
  \begin{enumerate}
  \item Monteverdi: Filtering $\rightarrow$ Change Detection
  \item Select the 2 images to process
    \begin{itemize}
    \item Raw images at 2 different dates
    \end{itemize}
  \item Make sure the {\em Use Change Detectors} check-box is activated
  \item Proceed as in the previous case
  \end{enumerate}
\end{frame}

\begin{frame}
  \frametitle{Hands On}
  \framesubtitle{Joint Classification with Features}
  \begin{enumerate}
  \item Monteverdi: Filtering $\rightarrow$ Change Detection
  \item Select the 2 images to process
    \begin{itemize}
    \item Feature images at 2 different dates
    \end{itemize}
  \item Proceed as in the previous cases
  \end{enumerate}
\end{frame}

\end{document}
