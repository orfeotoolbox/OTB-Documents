\documentclass[compress]{beamer}
\mode<presentation>
{
 \usetheme{Vilanova}
}

\usepackage[english]{babel}

\usepackage[latin1]{inputenc}

\usepackage{times}
\usepackage[T1]{fontenc}

\usepackage{amsfonts}
\usepackage{amsmath}
\usepackage{amssymb}
\usepackage{tikz}
%\usepackage{url}


\title{Pragmatic Remote Sensing}

\subtitle
{Change Detection} % (optional)

\author
{J. Inglada\inst{1} , E. Christophe\inst{2}}
\normalsize

\institute[Cnes, Crisp] % (optional, but mostly needed)
{\inst{1}\textsc{Centre d'�tudes Spatiales de la Biosph�re, Toulouse, France}
\and
\inst{2}\textsc{Centre for Remote Imaging, Sensing and Processing,\\ National University of Singapore}
}

\date{}

\pgfdeclareimage[height=96mm,width=128mm]{background}{fondsClairSansLogo}
\setbeamertemplate{background}{\pgfuseimage{background}}
\pgfdeclareimage[height=0.6cm]{logoIncrust}{logoIncrust}
\pgfdeclareimage[height=0.5cm]{logo_cesbio}{logo_cesbio}
\pgfdeclareimage[height=0.35cm]{logo_crisp}{logo_crisp}
\logo{
\begin{tabular}{lp{0.10\textwidth}lp{0.25\textwidth}r}
\href{http://www.cesbio.ups-tlse.fr/}{\pgfuseimage{logo_cesbio}}\href{http://www.crisp.nus.edu.sg/}{\pgfuseimage{logo_crisp}}
&&\footnotesize{IGARSS 2010, Honolulu}&&
\href{http://www.orfeo-toolbox.org}{\pgfuseimage{logoIncrust}}\\
\end{tabular}
}


\subject{Change detection in ORFEO Toolbox}


% Delete this, if you do not want the table of contents to pop up at
% the beginning of each subsection:
\AtBeginSubsection[]
{
  \begin{frame}<beamer>
    \frametitle{Outline}
    \tableofcontents[currentsection,currentsubsection]
  \end{frame}
}




% If you wish to uncover everything in a step-wise fashion, uncomment
% the following command: 

\beamerdefaultoverlayspecification{<+->}

\begin{document}

\begin{frame}
  \titlepage
{\tiny This content is provided under a Creative Commons
  Attribution 3.0 Unported License} \href{http://creativecommons.org/licenses/by/3.0/}{\includegraphics[width=0.05\textwidth]{../Ressources/CC-licence.png}}
\end{frame}


\begin{frame}
  \frametitle{Outline of the presentation}
  \tableofcontents[pausesections]
  % You might wish to add the option [pausesections]
\end{frame}


\section[Strategies]{Classical strategies for change detection}

\begin{frame}
  \frametitle{Possible approaches}
\begin{itemize}

\item{Strategy $1$: Simple detectors}

Produce an image of change likelihood
(by differences, ratios or any other approach) and thresholding it in
order to produce the change map.

\item{Strategy $2$: Post Classification Comparison}

Obtain two
land-use maps independently for each date and comparing them. 


\item{Strategy $3$: Joint classification}

Produce the change map directly from a joint
classification of both images.


\end{itemize}
\end{frame}


\section[Detectors]{Available detectors in OTB}
\begin{frame}
  \frametitle{Available detectors}
  \small
  \begin{itemize}
    \item Pixel-wise differencing of mean image values: 
      \begin{equation}
	I_{D}(i,j)=I_{2}(i,j)-I_{1}(i,j).
      \end{equation}
    \item Pixel-wise ratio of means:
      \begin{equation}
\displaystyle I_{R}(i,j) = 1 - min \left(\frac{\displaystyle I_{2}(i,j)}{\displaystyle I_{1}(i,j)},\frac{\displaystyle I_{1}(i,j)}{\displaystyle I_{2}(i,j)}\right).
\end{equation}
      \item Local correlation coefficient:
\begin{equation}
  I_\rho(i,j) = \frac{1}{N}\frac{\sum_{i,j}(I_1(i,j)-m_{I_1})(I_2(i,j)-m_{I_2})}{\sigma_{I_1}
\sigma_{I_2}}
\end{equation}
\item Kullback-Leibler distance between local distributions (mono and
  multi-scale)
  \item Mutual information (several implementations)
  \end{itemize}
  \normalsize
\end{frame}

\begin{frame}
  \frametitle{Hands On}
  \framesubtitle{Displaying differences}
  \begin{enumerate}
  \item Monteverdi: File $\rightarrow$ Concatenate Images
  \item Select the amplitudes of the 2 images to compare and build a 2
    band image
  \item Monteverdi: Visualization $\rightarrow$ Viewer
  \item Select the 2 band image just created
  \item In the {\em Setup} tab, select {\em RGB composition mode} and
    select, for instance, 1,2,2.
  \item Interpret the colors you observe
  \item The same thing could be done using feature images
  \end{enumerate}
\end{frame}

\begin{frame}
  \frametitle{Hands On}
  \framesubtitle{Thresholding differences}
  \begin{enumerate}
  \item Monteverdi: Filtering $\rightarrow$ Band Math
  \item Select the amplitudes of the 2 images to compare and compute a subtraction
  \item Monteverdi: Filtering $\rightarrow$ Threshold
  \item Select the difference image just created and play with the
    threshold value
  \item The same thing could be done using image ratios
  \item The same thing could be done using feature images
  \end{enumerate}
\end{frame}


\section[Application]{Interactive Change Detection}

\begin{frame}
  \frametitle{Interactive Change Detection}
  \begin{itemize}
    \item Generation of binary change maps using a GUI
    \item Uses simple change detectors as input
    \item The operator gives examples of the {\em change} and {\em no
    change} classes
    \item SVM learning and classification are applied
  \end{itemize}
\end{frame}


\begin{frame}
  \frametitle{Hands On}
  \framesubtitle{Joint Classification}
  \begin{enumerate}
  \item Monteverdi: Filtering $\rightarrow$ Change Detection
  \item Select the 2 images to process
    \begin{itemize}
    \item Raw images at 2 different dates
    \end{itemize}
  \item Uncheck {\em Use Change Detectors}
  \item Use the {\em Changed/Unchanged Class} buttons to select one of
    the classes
  \item Draw polygons on the images in order to give training samples
    to the algorithm
  \item {\em End Polygon} button is used to close the current polygon
  \item After selecting several polygons per class push {\em Learn}
  \item The {\em Display Results} button will show the result change detection
  \end{enumerate}
\end{frame}

\begin{frame}
  \frametitle{Hands On}
  \framesubtitle{Joint Classification with Change Detectors}
  \begin{enumerate}
  \item Monteverdi: Filtering $\rightarrow$ Change Detection
  \item Select the 2 images to process
    \begin{itemize}
    \item Raw images at 2 different dates
    \end{itemize}
  \item Make sure the {\em Use Change Detectors} check-box is activated
  \item Proceed as in the previous case
  \end{enumerate}
\end{frame}

\begin{frame}
  \frametitle{Hands On}
  \framesubtitle{Joint Classification with Features}
  \begin{enumerate}
  \item Monteverdi: Filtering $\rightarrow$ Change Detection
  \item Select the 2 images to process
    \begin{itemize}
    \item Feature images at 2 different dates
    \end{itemize}
  \item Proceed as in the previous cases
  \end{enumerate}
\end{frame}

\end{document}
