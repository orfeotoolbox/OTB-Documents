\documentclass[compress]{beamer}
\mode<presentation>
{
 \usetheme{Vilanova}
}

\usepackage[french]{babel}

\usepackage[utf8]{inputenc}

\usepackage{times}
\usepackage[T1]{fontenc}

\usepackage{amsfonts}
\usepackage{amsmath}
\usepackage{amssymb}
\usepackage{tikz}
\usepackage{eurosym}
%\usepackage{url}
\usepackage[normal]{subfigure}
\newcommand{\goodgap}{%
	\hspace{\subfigtopskip}%
	\hspace{\subfigbottomskip}}



%\newtheorem{definition}{Definition}

\title{Traitement d'images de télédétection}
\subtitle{Corrections radiométriques}




\author
{jordi.inglada@cesbio.cnes.fr}
\normalsize

\institute[Cesbio] % (optional, but mostly needed)
{\textsc{Centre d'Études Spatiales de la Biosphère, Toulouse, France}}

\date{}

\pgfdeclareimage[height=96mm,width=128mm]{background}{fondsClairSansLogo}
\setbeamertemplate{background}{\pgfuseimage{background}}
\pgfdeclareimage[height=0.6cm]{logoIncrust}{logoIncrust}
\pgfdeclareimage[height=0.5cm]{logo_cesbio}{logo_cesbio}
\logo{
\begin{tabular}{lp{0.25\textwidth}lp{0.25\textwidth}r}
\href{http://www.cesbio.ups-tlse.fr/}{\pgfuseimage{logo_cesbio}}
&&\footnotesize{AUF - Marrakech 2011}&&
\href{http://www.orfeo-toolbox.org}{\pgfuseimage{logoIncrust}}\\
\end{tabular}
}


\subject{Image radiometry in ORFEO Toolbox}




% Delete this, if you do not want the table of contents to pop up at
% the beginning of each subsection:
\AtBeginSubsection[]
{
  \begin{frame}<beamer>
    \frametitle{Outline}
    \tableofcontents[currentsection,currentsubsection]
  \end{frame}
}




% If you wish to uncover everything in a step-wise fashion, uncomment
% the following command: 

%\beamerdefaultoverlayspecification{<+->}
\begin{document}

\begin{frame}
  \titlepage
  \begin{center}
{\tiny Ce contenu est dérivé de la formation \href{http://www.orfeo-toolbox.org/packages/PragmaticRemoteSensing-handout.pdf}{``Pragmatic Remote
  Sensing''} dispensée par J. Inglada et E. Christophe en juillet 2010
  dans le cadre du colloque IGARSS. Il est mis à disposition selon les termes de la licence :\\
Creative Commons Paternité – Partage à l’Identique 3.0 non transcrit.} \href{http://creativecommons.org/licenses/by-sa/3.0/}{\includegraphics[width=0.05\textwidth]{/home/inglada/Dev/GH/IGARSS2010/Tutorial/Slides/Ressources/CC-licence.png}}    
  \end{center}
\end{frame}

\section*{Introduction}

\begin{frame}

  \frametitle{Introduction}
  \begin{block}{Objectifs}
   \begin{itemize}
   \item Obtenir des mesures physiques à partir des images
   \end{itemize}
  \end{block}
  \begin{block}{6S}
   \begin{itemize}
    \item Nous utilisons le code de transfert radiatif : http://6s.ltdri.org/
    \item Code bien testé et validé
    \item Traduit automatiquement de Fortran à C
    \item Encapsulation transparente dans l'OTB
   \end{itemize}
  \end{block}

\end{frame}

\section{Corrections radiométriques}
\begin{frame}

  \frametitle{Les corrections radiométriques en 4 étapes}



\begin{center}
\begin{tikzpicture}[scale=0.18]
   \tiny

    \draw[->,thick] (0,0) --  +(3,0);
%     \pause

    \draw[fill=black!30,rounded corners=2pt] (4,-2) rectangle +(6,4);
    \node[text width= 0.8cm] (SensorModel) at (7,0) {CN vers Lum};
%     \pause

    \draw[->,thick] (11,0) --  +(3,0);
%     \pause

    \draw[fill=black!30,rounded corners=2pt] (16,-2) rectangle +(6,4);
    \node[text width= 0.85cm] (SensorModel) at (19,0) {Lum vers Réfl};
%     \pause


    \draw[->,thick] (23,0) --  +(3,0);
%     \pause

    \draw[fill=black!30,rounded corners=2pt] (27,-2) rectangle +(6,4);
    \node[text width= 0.85cm] (SensorModel) at (30,0) {TOA vers TOC};
%     \pause

    \draw[->,thick] (34,0) --  +(3,0);
%     \pause

    \draw[fill=black!30,rounded corners=2pt] (38,-2) rectangle +(6.5,4);
    \node[text width= 0.85cm] (SensorModel) at (41,0) {Adjacence};
%     \pause

    \draw[->,thick] (45,0) --  +(3,0);

 \end{tikzpicture}
\end{center}

  \begin{block}{Enchaînement de filtres}
  Compatible avec la notion de {\em pipeline} de l'OTB
  \end{block}

\end{frame}

\subsection[CN vers Lum]{Du compte numérique vers la luminance}


\begin{frame}

  \frametitle{Du compte numérique vers la luminance}
     \begin{columns}

   \column{0.7\textwidth}
  \begin{block}{Objectif}
   \begin{itemize}
    \item Transformation du comte numérique en luminance
   \end{itemize}
  \end{block}
  Utilisation de otb::ImageToLuminanceImageFilter

  filterImageToLuminance->SetAlpha(alpha);

  filterImageToLuminance->SetBeta(beta);

  \column{0.3\textwidth}
  \footnotesize
  \begin{equation*}
   \mathbf{L_{TOA}^{k}} = \frac{ X^{k} } { \alpha_{k} } + \beta_{k}
  \end{equation*}
  \begin{itemize}
  \item $\mathbf{L_{TOA}^{k}}$ est la luminance incidente (en
  $W.m^{-2}.sr^{-1}.\mu m^{-1}$)
  \item $\mathbf{X^{k}}$  comte numérique
  \item $\alpha_{k}$ gain d'étalonnage pour la bande k
  \item $\beta_{k}$ biais d'étalonnage pour la bande k
  \end{itemize}

  \end{columns}
\end{frame}

\begin{frame}[fragile]
  \frametitle{Comment obtenir ces paramètres?}
  \begin{block}{Méta-données}
   \begin{itemize}
    \footnotesize
    \item Ces informations accompagnent souvent les images\ldots
    \item Mais le format des fichiers doit être connu!
   \end{itemize}
  \end{block}
  \begin{block}{A partir d'un fichier ASCII, ou à la main}
  \footnotesize
  \begin{verbatim}
  VectorType alpha(nbOfComponent);
  alpha.Fill(0);
  std::ifstream fin;
  fin.open(filename);
  double dalpha(0.);
  for( unsigned int i=0 ; i < nbOfComponent ; i++)
  {
      fin >> dalpha;
      alpha[i] = dalpha;
  }
  fin.close();
  \end{verbatim}
  \end{block}
\end{frame}

\subsection[Lum vers Réf]{De la luminance vers la réflectance}

\begin{frame}

  \frametitle{De la luminance vers la réflectance}
     \begin{columns}

   \column{0.6\textwidth}
  \begin{block}{Objectif}
   \begin{itemize}
    \item Transformer la luminance en réflectance
   \end{itemize}
  \end{block}
  \tiny
  Utilisation de otb::LuminanceToReflectanceImageFilter

  \texttt{filterLumToRef-> SetZenithalSolarAngle(zenithSolar);}

  \texttt{filterLumToRef-> SetDay(day);}

  \texttt{filterLumToRef-> SetMonth(month);}

  \texttt{filterLumToRef-> SetSolarIllumination(solarIllumination);}

  \column{0.3\textwidth}
  \footnotesize
  \begin{equation*}
   \rho_{TOA}^{k} = \frac{ \pi.\mathbf{L_{TOA}^{k}} } { E_{S}^{k}.cos(\theta_{S}).d/d_{0} }
  \end{equation*}
  \begin{itemize}
\tiny
  \item $\mathbf{rho_{TOA}^{k}}$ réflectance
  \item $\theta_{S}$ angle solaire zénithal
  \item $E_{S}^{k}$ éclairement solaire au sommet de l'atmosphère à
    une distance $d_{0}$ de la Terre
  \item $d/d_{0}$ rapport entre la distance Terre-Soleil au moment de
    l'acquisition par rapport à la moyenne
  \end{itemize}

  \end{columns}
\end{frame}


\subsection{ToA vers ToC}




\begin{frame}

  \frametitle{Du sommet de l'atmosphère au sol}

  \begin{block}{Objectif}
   \begin{itemize}
    \item Corriger les effets atmosphériques
   \end{itemize}
  \end{block}
 
  
  \begin{columns}
  \column{0.5\textwidth}
  \footnotesize
  \begin{equation*}
   \rho_{S}^{unif} = \frac{ \mathbf{A} }{ 1 + Sx\mathbf{A} }
  \end{equation*}
  \column{0.5\textwidth}
  \begin{equation*}
   \mathbf{A} = \frac{ \rho_{TOA} - \rho_{atm} }{ T(\mu_{S}).T(\mu_{V}).t_{g}^{all gas} }
  \end{equation*}
  \end{columns}
  \begin{itemize}
  \item $\rho_{TOA}$ réflectance au sommet de l'atmosphère
  \item $\rho_{S}^{unif}$ réflectance au sol sous hypothèse de surface
    lambertienne et environnement uniforme
  \item $\rho_{atm}$ réflectance intrinsèque de l'atmosphère
  \item $t_{g}^{all gas}$ albédo sphérique
  \item $T(\mu_{S})$ transmittance vers le bas
  \item $T(\mu_{V})$ transmittance vers le haut
  \end{itemize}
\end{frame}

\begin{frame}

  \frametitle{Du sommet de l'atmosphère au sol}
  \begin{itemize}
  \tiny
  \item Utilisation de \texttt{otb::ReflectanceToSurfaceReflectanceImageFilter}

  \texttt{filterToAtoToC->SetAtmosphericRadiativeTerms(correctionParameters);}

  \item \texttt{otb::AtmosphericCorrectionParametersTo6SAtmosphericRadiativeTerms} 

\texttt{parameters->SetSolarZenithalAngle();}

\texttt{parameters->SetSolarAzimutalAngle();}

\texttt{parameters->SetViewingZenithalAngle();}

\texttt{parameters->SetViewingAzimutalAngle();}

\texttt{parameters->SetMonth();}

\texttt{parameters->SetDay();}

\texttt{parameters->SetAtmosphericPressure();}

\texttt{parameters->SetWaterVaporAmount();}

\texttt{parameters->SetOzoneAmount();}

\texttt{parameters->SetAerosolModel();}

\texttt{parameters->SetAerosolOptical();}
\end{itemize}
\end{frame}


\subsection[Adjacence]{Effets d'adjacence}


\begin{frame}

  \frametitle{Effets d'adjacence}
     \begin{columns}

   \column{0.6\textwidth}
  \begin{block}{Objectif}
   \begin{itemize}
    \item Corriger les effets de voisinage
   \end{itemize}
  \end{block}

  \footnotesize
  Utilisation de \tiny \texttt{otb::SurfaceAdjacencyEffect6SCorrectionSchemeFilter}

  \footnotesize


  \tiny 
  \texttt{filterAdjacency->SetAtmosphericRadiativeTerms();}

  \texttt{filterAdjacency->SetZenithalViewingAngle();}

  \texttt{filterAdjacency->SetWindowRadius();}

  \texttt{filterAdjacency->SetPixelSpacingInKilometers();}

  \column{0.4\textwidth}
  \footnotesize
  \begin{equation*}
  \rho_{S} = \frac{ \rho_{S}^{unif}.T(\mu_{V}) - <\rho_{S}>.t_{d}(\mu_{v}) }{ exp(-\delta/\mu_{v}) }
  \end{equation*}
  \begin{itemize}
    \item $\rho_{S}^{unif}$ réflectance au sol sous hypothèse
      d'environnement uniforme
    \item $T(\mu_{V})$ transmittance vers le haut
    \item $t_{d}(\mu_{S})$ transmittance diffuse vers le haut
    \item $exp(-\delta/\mu_{v})$ transmittance directe vers le haut
    \item $<\rho_{S}>$ proportion de la contribution de l'environnement à
      la réflectance du pixel observé
  \end{itemize}

  \end{columns}
\end{frame}



\begin{frame}
\frametitle{La main à la pâte}
\begin{enumerate}
\item Monteverdi: Calibration $\rightarrow$ Optical calibration
\item Choisir une image
\item Regarder les paramètres extraits des méta-données
\item Appliquer la correction
\item Comparer les différentes valeurs obtenues pour un même pixel
\end{enumerate}
\end{frame}

\section{Fusion}

\begin{frame}
  \frametitle{Fusion}
  \framesubtitle{Ajouter du contenu spectral à une image à haute résolution}
\centering
\begin{columns}
\begin{column}{0.45\textwidth}
 \includegraphics[width=0.6\textwidth]{panSharp-pan-extract.jpg}\\
 \includegraphics[width=0.6\textwidth]{panSharp-xs-extract.jpg}
\end{column}
%\begin{column}{0.10\textwidth}
%$\Rightarrow$
%\end{column}
\begin{column}{0.45\textwidth}
 \includegraphics[width=0.6\textwidth]{panSharp-extract.jpg}
\end{column}
\end{columns}

\end{frame}

\begin{frame}
\frametitle{La main à la pâte}
\begin{enumerate}
\item Monteverdi: Ouvrir 2 images (Pan et XS) de même géométrie
\item Monteverdi: Filtering $\rightarrow$ Pan Sharpening
\end{enumerate}
\end{frame}

\end{document}
