\chapter{A brief tour of OTB-Applications}\label{chap:otb-applications}

\section{Introduction}\label{sec:appintro}

OTB-Applications is perhaps the older package of the \otb
suite after the OTB package itself. Since the \otb is a
library providing remote sensing functionalities, the only
applications that were distributed at the beginning were the examples
from the Software Guide and the tests. These applications are very
useful for the developer because their code is very short and only
demonstrates one functionality at a time, but in many cases a real
application would require combining together two or more functions
from the \otb, and providing a higher level interface to
handle parameters, input and output data and communication with the
user nicely.

The \app package was originally designed to provide applications
performing simple remote sensing tasks, more complex than simple
examples from the Software Guide, and with a more user-friendly
interface (either graphical or command-line), to demonstrate the use
of the \otb functions. The most popular applications are maybe the
\application{otbImageViewerManager}, which allows to open a collection
of images and navigate in them, and the
\application{otbSupervisedClassificationApplication}, which allows to
delineate training regions of interest on the image and classify the
image with a SVM classifier trained with these regions. During the
first 3 years of the \otb development, many more applications have
been added to this package, to perform various tasks. Most of them
come with a graphical user interface, apart from some small utilities
that are command-line.  For a complete list of these applications,
please refer to section~\ref{sec:appstruct}.

The development and release of the \mont software (see
chapter~\ref{chap:Monteverdi} at the end of year 2009 changed a lot of
things for the \app package: most of non-developer users were looking
for quite a long time for an applications providing \otb
functionalities under a unified graphical interface. Many applications
from the \app package were integrated to \mont as modules, and the
\app package lost a lot of its usefulness. No more applications were
added to the package and it was barely maintained, as new graphical
tools were directly embedded within \mont.

Then, some people started to regain interest in the \app package. \mont
is a great tool to perform numerous remote sensing and image
processing task in a minute, but it is not well adapted to heavier
(and longer) processing, scripting and batch processing. Therefore, in
2010 the \app package has been revamped: old applications have been
moved to a legacy folder for backward compatibility, and the
development team started to populate the package with compact
command-line tools to perform various heavy processing tasks. The
package is now rich of more than 40 tools, though not very well known
from the users for now. Although for now only the commmand-line
interface is fully functional, a work in progress aims at wrapping these
command-line tools to also offer QT graphical interfaces and integration
with the \qgis software as well as with other environment such as python,
IDL or Matlab (and with \mont).

\section{Installation}\label{sec:appinstall}

Detailed instruction on how to install the whole \otb suite either
from binary packages or from source are available in the \sg. Here, we
will focus only on the installation of the \app package from binary
package.

\subsection{Windows 2000/XP/Vista/Seven}
\label{ssec:app_windows_binaries}

For Windows 2000/XP/Vista/Seven users, an installation program exists
for OTB-Applications. This installers depends on dependencies that can
be installed through \osgeow. The packages that need to be installed
are \emph{gdal, curl, libtiff, libgeotiff, libjpeg, zlib and libpng}.

Remember that the corresponding dlls are to be accessible in the
system path. To ensure so, add the \osgeow bin directory to your
system path.

 Once the dependencies have been properly installed, please go the the
 \download, to get the installer. Double-click on the installer and
 let it guide you through the installation process.

\subsection{MacOS X}
\label{ssec:mac_binaries}

For now, no binary package is available for \app on MacOS X. You can
build the \app from sources by following instruction in the \sg.

\subsection{Linux}

\subsubsection{Ubuntu 10.04, 10.10 and 11.04}
\label{ssec:ubuntu_binaries}
For Unbuntu 10.04 (Lucid Lynx), 10.10 (Maverick Meerkat) and 11.04
(Natty Narwhal), the whole \otb suite is available through APT repositories.

If you are using apt command-line tools, use these command-lines to
add the otb repository to apt sources:
\begin{verbatim}
sudo aptitude install add-apt-repository 
sudo add-apt-repository ppa:otb/orfeotoolbox-stable
\end{verbatim}
Now run:
\begin{verbatim}
sudo aptitude update
sudo aptitude install otbapp
\end{verbatim}

If you are using \emph{Synaptic}, you can add the repositories, update
and install the packages through the graphical interface.

If you want to use OTB with bleeding edge versions of gdal and qgis,
there is an alternate UbuntuGIS repository.  You can add it by using
these command-lines:
\begin{verbatim}
sudo aptitude install add-apt-repository 
sudo apt-add-repository ppa:ubuntugis/ubuntugis-unstable
sudo add-apt-repository ppa:otb/orfeotoolbox-stable-ubuntugis
\end{verbatim}
Now run:
\begin{verbatim}
sudo aptitude update
sudo aptitude install otbapp
\end{verbatim}

Be careful not to add the two repositories, since they may cause
incompatibilities.

For further informations about ubuntu packages go to
href{https://launchpad.net/~otb/+archive/orfeotoolbox-stable}{orfeotoolbox-stable
  launchpad page} and click on \textbf{Read about installing}.

\textbf{apt-add-repository} will try to retrieve the GPG keys of the
repositories to certify the origin of the packages. If you are behind
a http proxy, this step won't work and apt-add-repository will stall
and eventually quit. You can temporarily ignore this error and proceed
with the update step. Following this, aptitude update will issue a
warning about a signature problem. This warning won't prevent you from
installing the packages.

\subsubsection{OpenSuse 11.2 and higher}
\label{ssec:opensuse_binaries}

For OpenSuse 11.2 and higher, the whole \otb suite is available
through \emph{zypper}.

First, you need to add the appropriate repositories with these
command-lines (please replace $11.4$ by your OpenSuse version):
\begin{verbatim}
sudo zypper ar 
http://download.opensuse.org/repositories/games/openSUSE_11.4/ Games
sudo zypper ar 
http://download.opensuse.org/repositories/Application:/Geo/openSUSE_11.4/ GEO
sudo zypper ar 
http://download.opensuse.org/repositories/home:/tzotsos/openSUSE_11.4/ tzotsos
\end{verbatim}

Now run:
\begin{verbatim}
sudo zypper refresh
sudo zypper install Orfeo-Applications
\end{verbatim}

Alternatively you can use the One-Click Installer from the
\href{http://software.opensuse.org/search?q=Orfeo&baseproject=openSUSE\%3A11.4&lang=en&include_home=true&exclude_debug=true}{openSUSE
  Download page} or add the above repositories and install through
Yast Package Management.

In case you wish to test the latest version of the packages, you can run:
\begin{verbatim}
sudo zypper refresh
sudo zypper install Orfeo-Applications-beta
\end{verbatim}

\section{Structure of the package}\label{sec:appstruct}
This section will provide the list of the applications available with details about command line arguments... 
