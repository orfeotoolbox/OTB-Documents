\documentclass[10pt,a4paper]{article}

\usepackage[utf8]{inputenc}
\usepackage[english]{babel}
\usepackage[T1]{fontenc}
\usepackage{amsmath}
\usepackage{amsfonts}
\usepackage{amssymb}
\usepackage{graphicx}
\usepackage{lmodern}
\usepackage{hyperref}
\usepackage[left=2.7cm,right=2.7cm,top=2cm,bottom=3cm]{geometry}
\usepackage{verbatim}
\usepackage{xcolor}
\usepackage{url}

\hypersetup{
    colorlinks,
    linkcolor={red!50!black},
    citecolor={blue!50!black},
    urlcolor={blue!80!black}
}

\title{\includegraphics[scale=1]{Art/otb-logo.png}\\
  OTB Installation Guide\\
  {\small\url{https://www.orfeo-toolbox.org}}
}

\begin{document}

\maketitle

\tableofcontents

\clearpage
\section{Windows}

\subsection{QGIS}
Install QGIS: \url{http://www.qgis.org/en/site/forusers/download.html}.

\subsection{Command line}
Install a minimal command line environment for Windows. For instance, Git for Windows is very easy to install:
\url{http://git-scm.com/download/win}.

\subsection{OTB and Monteverdi}
To install OTB and Monteverdi, download the appropriate package for your architecture (32 or 64 bits). If you have a 32 bit machine:

\begin{verbatim}
OTB-6.0.0-win32.zip
\end{verbatim}

If you have a 64 bit machine:

\begin{verbatim}
OTB-6.0.0-win64.zip
\end{verbatim}

These files are available at:
\url{https://www.orfeo-toolbox.org/download/}.
Extract the zip archive in your personal folder, for instance in:\\
\begin{centering}
\texttt{C:{\textbackslash}Users{\textbackslash}John{\textbackslash}install{\textbackslash}}.
\end{centering}

\subsection{Test the installation}
Once the installation is done, OTB applications can be used in several ways. Check that you have a working OTB with the following steps:
\begin{enumerate}

\item Launch \texttt{monteverdi.bat} from the installation folder.

\item Try to open a tif image in Monteverdi (see
figure~\ref{fig:monteverdi}). A demonstration tif image is available here: \url{https://git.orfeo-toolbox.org/otb-data.git/blob/HEAD:/Examples/QB\_Toulouse\_Ortho\_PAN.tif}.

\item The application browser is available from the ``View'' menu 
$\rightarrow$ "OTB-Applications browser".
(see figure \ref{fig:windows-mapla}).

\item Go to the \texttt{bin} folder in the OTB install directory and double-click on the \texttt{.bat} file corresponding to the application to be run, for instance:\\
\texttt{John{\textbackslash}install{\textbackslash}OTB-6.0.0-win32{\textbackslash}bin{\textbackslash}otbgui\_Rescale.bat}
(see figure \ref{fig:windows-otbgui}).

\end{enumerate}

\begin{figure}[h]
  \center
  \includegraphics[width=1\textwidth]{Art/monteverdi-tif.png}
  \caption[]{Monteverdi}
  \label{fig:monteverdi}
\end{figure}

\begin{figure}[h]
  \center
  \includegraphics[scale=1]{Art/windows-mapla.png}
  \caption[]{OTB applications can be used from Monteverdi}
  \label{fig:windows-mapla}
\end{figure}

\begin{figure}[h]
  \center
  \includegraphics[scale=1]{Art/windows-otbgui.png}
  \caption[]{OTB Graphical User Interface}
  \label{fig:windows-otbgui}
\end{figure}

% En commentaire pour l'instant, c'est peut être un peu inutile puisqu'il n'y a
% aucune action associée en terme d'installation
\begin{comment}
\subsection{OTB depuis QGIS}

Les applications OTB sont disponibles depuis QGIS. Pour cela

Attention ! QGIS est livré avec ses propres binaires OTB. Les applications ainsi
ouvertes depuis QGIS sont souvent différentes de celles installée ci-dessus, car
il s'agit typiquement d'une version plus ancienne.
Éventuellement, il est possible de remplacer l'OTB livrée avec QGIS en supprimant
ou écrasant les fichiers dans \texttt{}.

\begin{figure}[h]
  \center
  \includegraphics[width=1\textwidth]{Art/qgis-otb.png}
  \caption[]{Intégration QGIS - OTB}
  \label{fig:otb-qgis}
\end{figure}
\end{comment}

\clearpage
\section{Linux (Ubuntu example)}

This section explains how to install the tools in a Linux environment like Ubuntu. This procedure can also be used on other Linux distributions by using the appropriate package manager (dnf, yum,
emerge, pacman\ldots)

\subsection{QGis}
QGis can be installed via a command line like:
\begin{verbatim}
sudo apt-get install qgis
\end{verbatim}

\subsection{Dependencies installation}
Before installing the self-extracting Linux binary for OTB, several system dependencies have to be installed. In a terminal, type the following:
\begin{verbatim}
sudo apt-get install libx11-6 libxext6 libxau6 libxxf86vm1 libxdmcp6 libdrm2
\end{verbatim}
or the equivalent for other distributions.

You will need the libgl1 and libglu1 libraries, which have different implementations (MESA, FGLRX, NVIDIA, etc.). If you don't already have these libraries, you can use the MESA implementation:
\begin{verbatim}
sudo apt-get install libgl1-mesa-glx libglu1-mesa
\end{verbatim}

\subsection{Install OTB and Monteverdi}
Download the self-extracting binary for Linux (64 bits), available at:
\begin{center}
\url{https://www.orfeo-toolbox.org/download}
\end{center}

The self-extracting binary will extract itself in the current directory. First, the archive has to be made executable, and then it can be run:
\begin{verbatim}
chmod +x OTB-6.0.0-Linux64.run
./OTB-6.0.0-Linux64.run
\end{verbatim}

The executable binaries will be inside the 'bin' directory, and you can put this directory in your PATH variable if you want to. 

There are also scripts which set all the environment variables to allow to run Monteverdi and Mapla:
\begin{verbatim}
cd OTB-6.0.0-Linux64
./monteverdi.sh
./mapla.sh
\end{verbatim}

\subsection{Test the installation}
Once the installation is done, OTB applications can be run in several ways. Check that you have a working installation with the following steps:
\begin{enumerate}

\item Try to open a tif image in Monteverdi (see
figure~\ref{fig:monteverdi}). A demonstration tif image is available here: \url{https://git.orfeo-toolbox.org/otb-data.git/blob/HEAD:/Examples/QB\_Toulouse\_Ortho\_PAN.tif}.

\item The application browser is available from the ``View'' menu 
$\rightarrow$ "OTB-Applications browser".
(see figure \ref{fig:windows-mapla}).

\item Run an application using the terminal, for instance
\texttt{otbgui\_Rescale}. (see figure \ref{fig:windows-otbgui}).

\end{enumerate}

\clearpage
\section{Mac OS X}

The software can also be installed on Mac OS X.

First install QGIS using the instructions for Mac OS X on the official site. To install OTB and  Monteverdi download 
OTB-6.0.0-Darwin64.run from \url{https://www.orfeo-toolbox.org/download/}.

This is a self-extracting binary which extracts itself in the current directory. Testing the installation is done like in the Linux case.

\end{document}
