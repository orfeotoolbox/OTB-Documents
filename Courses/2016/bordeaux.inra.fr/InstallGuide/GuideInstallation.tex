\documentclass[12pt,a4paper]{article}

\usepackage[utf8]{inputenc}
\usepackage[francais]{babel}
\usepackage[T1]{fontenc}
\usepackage{amsmath}
\usepackage{amsfonts}
\usepackage{amssymb}
\usepackage{graphicx}
\usepackage{lmodern}
\usepackage{hyperref}
\usepackage[left=2cm,right=2cm,top=2cm,bottom=2cm]{geometry}

\title{Guide d'installation OTB}

\begin{document}

\maketitle

\section{Windows}

\subsection{QGIS}
Installer QGIS: \url{http://www.qgis.org/fr/site/forusers/download.html}.

\subsection{Ligne de commande}
Installer un environement minimal de ligne de commande pour Windows. Par exemple
Git for Windows est assez facile à installer:
\url{http://git-scm.com/download/win}.

\subsection{OTB et Monteverdi}

Pour installer OTB 5.2.1 et Monteverdi 3.0.1, telecharger les paquets
correspondants à votre architecture (32 ou 64 bits). Si votre machine est 32
bits:
\begin{verbatim}
Monteverdi-3.0.1-win32.zip
OTB-5.2.1-win32.zip
\end{verbatim}

Sinon si votre machine est 64 bits:
\begin{verbatim}
Monteverdi-3.0.1-win64.zip
OTB-5.2.1-win64.zip
\end{verbatim}

Ces archives sont disponibles sur \url{https://www.orfeo-toolbox.org/packages/}.
Extraire les deux archives zip dans votre repertoire personel, par exemple dans \texttt{C:{\textbackslash}Utilisateurs{\textbackslash}Martin{\textbackslash}OTB{\textbackslash}}.

\subsection{Lier avec QGIS}

\subsection{Tester l'installation}

\clearpage
\section{Ubuntu}

\end{document}
