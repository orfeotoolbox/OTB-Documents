%----------------------------------------------------------------------------------------
%	PACKAGES & THEMES
%----------------------------------------------------------------------------------------

\documentclass[8pt]{beamer}

\usepackage{etex}
\mode<presentation> {

\usetheme{Vilanova}
}

\usepackage[english]{babel}
\usepackage[utf8]{inputenc}
\usepackage{array}
\usepackage{chronology}
\let\CHRONOLOGY\chronology
\let\endCHRONOLOGY\endchronology
\def\chronology{\shorthandoff{;}\CHRONOLOGY}
\def\endchronology{\endCHRONOLOGY\shorthandon{;}}
\usepackage{pstricks}
\usepackage{graphicx}
\usepackage{booktabs}
\usepackage{amsmath,amssymb,amsthm}
\usepackage{xcolor}
\usepackage{textpos}
\usepackage{tikz}
\usepackage{xmpincl}
\usetikzlibrary{arrows}
\usepackage{pifont}

\usepackage{listings,color}

\definecolor{listcomment}{rgb}{0.0,0.5,0.0}
\definecolor{listkeyword}{rgb}{0.0,0.0,0.5}
\definecolor{listnumbers}{gray}{0.65}
\definecolor{listlightgray}{gray}{0.955}
\definecolor{listwhite}{gray}{1.0}


%% \setbeamertemplate{background canvas}{\includegraphics
%%    [width=\paperwidth,height=\paperheight]{./images/title.pdf}}

\AtBeginSection[]
{
\addtocounter{framenumber}{-1}
\begin{frame}
\frametitle{Sommaire}
\tableofcontents[currentsection]
\end{frame}}

%----------------------------------------------------------------------------------------
%	PAGE TITRE
%----------------------------------------------------------------------------------------
\title{Orfeo ToolBox users meeting and hackfest 2015}
\includexmp{images/cc}
\subtitle{Third parties policy and SuperBuild}
\author{OTB development team}% date and event here
\date{3 - 5 june 2015, Toulouse}

\pgfdeclareimage[height=96mm,width=128mm]{background}{images/fondsClairSansLogo}
\pgfdeclareimage[height=0.2cm]{cc}{images/CC-licence.png}
\setbeamertemplate{background}{\pgfuseimage{background}}
\pgfdeclareimage[height=0.6cm]{logoIncrust}{images/logoIncrust}
\logo{
\begin{tabular}{p{0.22\textwidth}p{0.58\textwidth}p{0.1\textwidth}p{0.1\textwidth}}
\href{http://creativecommons.org/licenses/by-sa/3.0/}{\pgfuseimage{cc}}
& \vspace{-0.03\textwidth} \scriptsize{} % date and event here
&  & \href{http://www.orfeo-toolbox.org}{\pgfuseimage{logoIncrust}}\\
\end{tabular}
}

\begin{document}
\begin{frame}
\titlepage
\end{frame}

\begin{frame}
\frametitle{Introduction}

\begin{itemize}
\item Principle : give a mathematical expression to be applied to an image
\item The BandMathX image filter or application will parse this expression
\item ... and will apply it to each pixel, or neighborhood of pixels automatically !
\item Based on MuParseX (Ingo Berg).
\end{itemize} 

\end{frame}

\begin{frame}
\frametitle{Variables}

\begin{itemize}
\item im1 =  a pixel from first input, made of n components (n bands) = Vector
\item im1bj = jth component of a pixel from first input (first band is indexed by 1) = Scalar
\item im1PhyX and im1PhyY = spacing of first input in X and Y directions (horizontal and vertical) = Scalar 
\item idxX and idxY = represent the indices of the current pixel (scalars) = Scalar
\item im1bjMean  im1bjMin  im1bjMax  im1bjSum  im1bjVar  = mean,  min,  max,  sum,  variance  of  jth band from first input (global statistics) = Scalar
\item im1bjNkxp = a neighbourhood (’N’) of pixels of the jth component from first input, of size kxp = Matrix 
\end{itemize}

\begin{center} 
\begin{tabular}{|c|c|c|}
\hline
.	& .	& . \\
\hline
.	& .	& . \\
\hline
.	& .	& . \\
\hline
.	& .	& . \\
\hline
.	& .	& . \\
\hline
\end{tabular}
\end{center}
\begin{center} 
Neighborhood of 3x5. k/p = horizontal/vertical direction. k and p must be odd numbers.
\end{center}

\end{frame}


\begin{frame}
\frametitle{Some examples \#1}

\begin{itemize}
\item Always keep in mind that a pixel of an otb::VectorImage is always represented as a row vector inside the muParserX framework
\item MuParserX only addresses mathematically well-defined formulas
\end{itemize}

\begin{center}
\begin{tabular}{c | c}
Formula & Status \\
\hline \\
im1 + im2 & correct only if the two first inputs have the same number of bands (*)\\
im1 + 1  & incorrect even if im1 represents a one-band pixel  \\
im1 + \{1\} & much better ! \\
im1 + \{1,1,1,...,1\} & correct if im1 is made of n bands \\

\end{tabular}
\end{center}

(*) Note : batch mode



\end{frame}

\begin{frame}
\frametitle{Some examples \#2}


\begin{itemize}
\item Always keep in mind that a pixel of an otb::VectorImage is always represented as a row vector inside the muParserX framework
\item MuParserX only addresses mathematically well-defined formulas
\end{itemize}


\begin{center}
\begin{tabular}{c | c}
Formula & Status \\
\hline \\
im1b1 + 1 & correct  \\
\{im1b1\} + \{1\} & correct  \\
im1b1 + \{1\} & incorrect \\
\{im1b1\} + 1 & incorrect \\
im1 + \{im2b1,im2b2\}  &  correct if im1 represents a pixel of two components \\

\end{tabular}
\end{center}


\end{frame}


\begin{frame}
\frametitle{Some examples \#3}


\begin{itemize}
\item Always keep in mind that a pixel of an otb::VectorImage is always represented as a row vector inside the muParserX framework
\item MuParserX only addresses mathematically well-defined formulas
\end{itemize}


\begin{center}
\begin{tabular}{c | c}
Formula & Status \\
\hline \\
\{im2b1,im2b2\}*\{1,2\} & incorrect  \\
\{im2b1,im2b2\}*\{1,2\}' & correct  \\
im2*\{1,2\}' & correct if im2 represents a pixel of two components \\

\end{tabular}
\end{center}


\end{frame}


\begin{frame}
\frametitle{New operators and functions}

New operators and functions have been implemented within BandMathX application. These ones can be divided into two categories.

\begin{itemize}
\item adaptation of existing operators/functions, that were not originally defined for vectors and matrices (for instance cos, sin, ...). These new operators/ functions keep the original names to which we add the prefix ”v” for vector (vcos, vsin, ...). 
\item truly new operators/functions.
\end{itemize}
\end{frame}


\begin{frame}
\frametitle{New operators and functions}


\begin{itemize}
\item div (element-wise division) and dv (division by a scalar)
\item mult (element-wise multiplication) and mlt (multiplication by a scalar)
\item pow (element-wise exponentiation) and w (exponentiation by a scalar)
\end{itemize}


\begin{center}
\begin{tabular}{c | c | c}
Operator/function & ex. 1 & ex. 2 \\
\hline \\
div and dv & im1 ~ div ~ im2 &  m1 ~ dv ~ 2.0 \\
mult and mlt & im1 ~  mult ~ im2 & im1 ~  mlt ~ 2.0  \\
pow and pw & im1 ~ pow ~ im2 & im1 ~ pw ~ 2.0
\end{tabular}
\end{center}


\end{frame}


\begin{frame}
\frametitle{New operators and functions}


\begin{itemize}
\item dotpr : This function allows the dot product between two vectors or matrices (actually in our case, a kernel and a neighbourhood of pixels) : $\sum_{(i,j)} m_1(i,j)*m_2(i,j)$


\item For instance: dotpr(kernel1,im1b1N3x5) is correct provided that kernel1 and im1b1N3x5 have the same dimensions. 
\item The function can take as many neighbourhoods as needed in inputs. Thus, if n neighbourhoods must be processed, the output will consist in a row vector of n values. This behaviour is typical of the functions implemented in the BandMathX application.
\end{itemize}

\end{frame}


\begin{frame}
\frametitle{New operators and functions}


\begin{itemize}
\item mean   : mean value of a given vector or neighborhood 
\item var    : variance value of a given vector or neighborhood 
\item median : median value of a given vector or neighborhood 
\item corr   : correlation between two vectors or matrices of the same dimensions (the function takes two inputs)
\item maj    : compute the most represented element within a vector or a matrix 
\item vmin and vmax : min or max value of a given vector or neighborhood 
\end{itemize}


\begin{center}
\begin{tabular}{c | c }
Operator/function & example \\
\hline \\
mean (*) & mean(im1b1N3x3,im1b2N3x3,im1b3N3x3,im1b4N3x3) \\
var (*) & var(im1b1N3x3) \\
median (*) & median(im1b1N3x3) \\
corr (two inputs) & corr(im1b1N3x3,im1b2N3x3) \\
maj (*) & maj(im1b1N3x3,im1b2N3x3) \\
vmin and vmax (one input) & (vmax(im3b1N3x5)+vmin(im3b1N3x5)) ~ div ~ \{2.0\} \\
\end{tabular}
\end{center}

(*) : the function can take as many inputs as needed; one mean value is computed per input
\end{frame}

\begin{frame}
\frametitle{New operators and functions}


\begin{itemize}
\item cat    : This function allows to concatenate the results of several expressions into a multidimensional vector, whatever their respective dimensions (the function can take
as many inputs as needed)
\item band   : This function allows to select specific bands from an image, and/or to rearrange them in a new vector.
\end{itemize}


\begin{center}
\begin{tabular}{c | c }
Operator/function & example \\
\hline \\
cat & cat(im3b1,vmin(im3b1N3x5),median(im3b1N3x5),vmax(im3b1N3x5)) \\
bands & bands(im1,\{1,2,1,1\}) \\
vect2scal & vect2scal({0.5}) \\
\end{tabular}
\end{center}

Note about cat function : the user should prefer the use of semi-colons (;) when setting expressions, instead of directly use this function.
The application will call the function 'cat' automatically.
\end{frame}


\begin{frame}
\frametitle{Filter : example 1}


\begin{itemize}
\item \#include "otbBandMathXImageFilter.h"
\item ....
\item typedef otb::BandMathXImageFilter<ImageType>  FilterType;
\item ...
\item FilterType::Pointer filter = FilterType::New();
\item ...
\item filter$->$SetExpression("im1-mean(im1b1N5x5,im1b2N5x5,im1b3N5x5,im1b4N5x5)");
\item filter$->$SetNthInput(0,reader$->$GetOutput()); ou filter$->$SetNthInput(0, reader$->$GetOutput(),"imageA");
\item writer$->$SetInput(filter$->$GetOutput()); 
\item writer$->$Update();
\end{itemize}

\end{frame}


\begin{frame}
\frametitle{Filter : example 2}


\begin{itemize}
\item filter$->$SetMatrix("kernel","\{ 0.1 , 0.1 , 0.1; 0.1 , 0.2 , 0.1; 0.1 , 0.1 , 0.1 \}");
\item filter$->$SetConstant("cst",1.0);
\item filter$->$SetExpression("bands(im1,\{1,2,3\})-dotpr(kernel,im1b1N3x3,im1b2N3x3,im1b3N3x3) + \{cst,cst,cst\}");  

\end{itemize}

Note : concatenation of the results of several expressions into a multidimensional vector is possible.
For this purpose, use semi-colons (;) as separators between expressions.

\end{frame}




\begin{frame}
\frametitle{Filter : example 3}

\begin{itemize}
\item filter$->$ExportContext(argv[4]);
\item filter$->$ImportContext(argv[4]);
\end{itemize}

\vspace{1cm}

\fbox{\begin{minipage}{\textwidth}
\#F cst 1.1
\newline
\#M kernel1 \{ 0.1 , 0.2 , 0.3; 0.4 , 0.5 , 0.6; 0.7 , 0.8 , 0.9; 1 , 1.1 , 1.2; 1.3 , 1.4 , 1.5\}
\newline
\#E dotpr(kernel1,imageAb1N3x5,imageAb2N3x5) ; im2b1 * cst
\end{minipage}}



\end{frame}

\begin{frame}
\frametitle{Application}

\includegraphics[width=7cm,height=7cm]{images/bandmathx.png}

\end{frame}


\begin{frame}
\frametitle{Real life examples}

\begin{itemize}
\item Conversion 12 bits RGBNIR to 8 bits RGB with gamma correction : (im1 - min) div ( (max-min) pw gamma )
\item Many vegetation indices into a single image : im1b4/im1b1 ; (im1b4-im1b1) / (im1b4+im1b1) ; (alpha*im1b4-im1b1) / (alpha*im1b4+im1b1)
\item Single scale retinex : vlog(im1) - vlog( mean(im1b1N5x5,im1b2N5x5,im1b3N5x5,im1b4N5x5) )
\newline
\newline
\item Spectral angle : vacos( dotpr(im1,im2) div (vnorm(im1) mult vnorm(im2)) )
\item Maps of local maxima/minima : (im1b1 $>$ vect2scal(vmax(im1b1N5x5)) )?\{1\}:\{0\}
\end{itemize} 

\end{frame}


\end{document}
