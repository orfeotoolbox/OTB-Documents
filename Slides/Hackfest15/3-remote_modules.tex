%----------------------------------------------------------------------------------------
%	PACKAGES & THEMES
%----------------------------------------------------------------------------------------

\documentclass[8pt]{beamer}

\usepackage{etex}
\mode<presentation> {

\usetheme{Vilanova}
}



\usepackage[english]{babel}
\usepackage[utf8]{inputenc}
\usepackage{array}
\usepackage{chronology}
\let\CHRONOLOGY\chronology
\let\endCHRONOLOGY\endchronology
\def\chronology{\shorthandoff{;}\CHRONOLOGY}
\def\endchronology{\endCHRONOLOGY\shorthandon{;}}
\usepackage{pstricks}
\usepackage{graphicx}
\usepackage{booktabs}
\usepackage{amsmath,amssymb,amsthm}
\usepackage{xcolor}
\usepackage{textpos}
\usepackage{tikz}
\usepackage{xmpincl}
\usetikzlibrary{arrows}
\usepackage{pifont}

\usepackage{listings,color}

\definecolor{listcomment}{rgb}{0.0,0.5,0.0}
\definecolor{listkeyword}{rgb}{0.0,0.0,0.5}
\definecolor{listnumbers}{gray}{0.65}
\definecolor{listlightgray}{gray}{0.955}
\definecolor{listwhite}{gray}{1.0}


%% \setbeamertemplate{background canvas}{\includegraphics
%%    [width=\paperwidth,height=\paperheight]{./images/title.pdf}}

\AtBeginSection[]
{
\addtocounter{framenumber}{-1}
\begin{frame}
\frametitle{Sommaire}
\tableofcontents[currentsection]
\end{frame}}

%----------------------------------------------------------------------------------------
%	PAGE TITRE
%----------------------------------------------------------------------------------------
\title{Orfeo ToolBox users meeting and hackfest 2015}
\includexmp{images/cc}
\subtitle{Contributions through remote modules}
\author{OTB development team}% date and event here
\date{3 - 5 june 2015, Toulouse}

\pgfdeclareimage[height=96mm,width=128mm]{background}{images/fondsClairSansLogo}
\pgfdeclareimage[height=0.2cm]{cc}{images/CC-licence.png}
\setbeamertemplate{background}{\pgfuseimage{background}}
\pgfdeclareimage[height=0.6cm]{logoIncrust}{images/logoIncrust}
\logo{
\begin{tabular}{p{0.22\textwidth}p{0.58\textwidth}p{0.1\textwidth}p{0.1\textwidth}}
\href{http://creativecommons.org/licenses/by-sa/3.0/}{\pgfuseimage{cc}}
& \vspace{-0.03\textwidth} \scriptsize{} % date and event here
&  & \href{http://www.orfeo-toolbox.org}{\pgfuseimage{logoIncrust}}\\
\end{tabular}
}

\begin{document}
\begin{frame}
\titlepage
\end{frame}

\begin{frame}
\frametitle{By now, you know that:}
\begin{itemize}
\item Orfeo ToolBox is now organized in groups/modules
\item Each module contains classes, applications, third parties, and their related tests
\item Each module can be deactivated
\item There is a dependency tracking mechanism that deactivate downstream modules
\item CMake automagically recognize any subfolder of Modules/*/ as a module
\item In particular, we have a dedicated Modules/Remote folder to host contributions
\end{itemize}

\textcolor{red}{So, next time you want to contribute a piece of code, why not write a module?}


\end{frame}

\begin{frame}[fragile]
\frametitle{Writing a remote module is simple (1/2)}
Let's write a remote module to share the \textbf{Awesome} algorithm code:
\begin{itemize}
\item A header file: \textit{otbAwesomeFilter.h}
\item A template methods file: \textit{otbAwesomeFilter.txx}
\item A source file: \textit{otbAwesomeFilter.cxx}
\end{itemize}

\begin{block}{otb-module.cmake}
\begin{small}
\begin{verbatim}
otb_module(Awesome 
           DEPENDS OTBImageBase OTBStatistics 
           DESCRIPTION "Implementation of awesome algorithm")
\end{verbatim}
\end{small}
\end{block}

\begin{block}{CMakeLists.txt}
\begin{small}
\begin{verbatim}
project(Awesome)
set(Awesome_LIBRARIES OTBAwesome) # only if your module contain a src folder
otb_module_impl()
\end{verbatim}
\end{small}
\end{block}
\end{frame}

\begin{frame}[fragile]
\frametitle{Writing a remote module is simple (2/2)}
\begin{block}{The include folder}
\begin{itemize}
\item There goes the \textit{otbAwesomeFilter.h} and \textit{otbAwesomeFilter.txx} files
\item A nothing more!
\end{itemize}
\end{block}

\begin{block}{The src folder (not needed if template only code)}
\begin{itemize}
\item There goes the \textit{otbAwesomeFilter.cxx} file
\item And a \textit{CMakeLists.txt} file
\end{itemize}
\end{block}


\begin{block}{CMakeLists.txt in src folder}
\begin{small}
\begin{verbatim}
set(OTBAwesome_SRC otbAwesomeFilter.cxx)

add_library(OTBAwesome ${OTBAwesome_SRC})

target_link_libraries(OTBAwesome ${OTBImageBase_LIBRARIES} ${OTBStatistics_LIBRARIES})

otb_module_target(OTBAwesome)
\end{verbatim}
\end{small}
\end{block}

\textcolor{red}{That'all folks! (extra folders and cmake required for tests and app)}

\end{frame}

\begin{frame}[fragile]
\frametitle{Great, but how do I build it?}
\begin{block}{Building the new module}
\begin{enumerate}
\item Place your module in Modules/Remote folder
\item Run cmake configuration (ensure the BUILD\_DEFAULT\_MODULES is ON, or your module is enabled)
\item make
\end{enumerate}
\end{block}

\begin{block}{Sharing the new module}
\begin{enumerate}
\item Place your code on a public git repository
\item Others can test it directly by cloning it in Modules/Remote
\item But you can also provide us a awesome.remote.cmake file for inclusion in OTB sources (in Modules/Remote). Your module will be visible at configuration time.
\end{enumerate}
\begin{center}
\begin{small}
\begin{verbatim}
#Contact: OTB Guru <og@orfeo-toolbox.org>

otb_fetch_module(Awesome
  "A truly awesome filter"
  GIT_REPOSITORY www.somegitrepo.org/awesome.git
  GIT_TAG 00000000000)
\end{verbatim}
\end{small}
\end{center}
\end{block}

\end{frame}

\begin{frame}
\frametitle{Benefits of remote modules}
\begin{itemize}
\item Writing new code for OTB is much, much simpler
\item You do not need to modify a single line of OTB code to share contributions with others
\item You remain the manager of the contributed code, and your contribution is clearly attributed
\item You can licence your contribution with whatever licence you want
\item If you are accepted as an official remote module (with *.remote.cmake in OTB sources), you benefit from dashboard and packaging
\item If you do not want to maintain the new code, or if you feel like it, your remote module can become an official internal OTB module
\end{itemize}
\end{frame}

\begin{frame}
\frametitle{Official remote modules acceptance policy}
\begin{itemize}
\item Remote module source code should be hosted on a publicly available Git repository
\item Author of the remote module should be identified and registed to otb-developers mailing list
\item Author of the remote module accepts to be contacted by developers or users regarding issues with his module (on a best effort basis),
\item Remote module source code should comply with OTB style as much as possible,
\item Remote module source code should be documented using doxygen tags,
\item Remote module should provide a minimal set of tests to ensure build of template code and basic non-regression testing,
\item Remote module should come with a form of documentation (website, publication, readme file ...)
\item Remote module should not embed code from any third party software (unless strong arguments are given by the author, in which case an exception can be made),
\item Remote module should avoid depending on new third parties if possible,
\item Remote module author should be the copyright owner and comply with licences of any third party,
\item Author of remote module should provide a small description of the remote module to be added on OTB website.
\end{itemize}
\end{frame}

\begin{frame}
\frametitle{Official remote modules release policy}

During the OTB release process, a remote module will be included in source and binary packages if
\begin{itemize}
\item Dashboard submission exist and show that the remote module:
     \begin{itemize}
       \item Builds on all plateform,
       \item Passes all tests on the reference platform,
       \item Does not have any test crashing (i.e. failing with core dump or memory issues) on remaining platform.
     \end{itemize}
\item The remote module complies with the remote module acceptance policy at the time of the Release Candidate 
\end{itemize}

\end{frame}

\end{document}
