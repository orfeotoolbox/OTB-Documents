%----------------------------------------------------------------------------------------
%	PACKAGES & THEMES
%----------------------------------------------------------------------------------------

\documentclass[8pt]{beamer}

\usepackage{etex}
\mode<presentation> {

\usetheme{Vilanova}
}



\usepackage[english]{babel}
\usepackage[utf8]{inputenc}
\usepackage{array}
\usepackage{chronology}
\let\CHRONOLOGY\chronology
\let\endCHRONOLOGY\endchronology
\def\chronology{\shorthandoff{;}\CHRONOLOGY}
\def\endchronology{\endCHRONOLOGY\shorthandon{;}}
\usepackage{pstricks}
\usepackage{graphicx}
\usepackage{booktabs}
\usepackage{amsmath,amssymb,amsthm}
\usepackage{xcolor}
\usepackage{listings}
\lstset{basicstyle=\ttfamily,
  showstringspaces=false,
  commentstyle=\color{red},
  keywordstyle=\color{blue}
}
\usepackage{textpos}
\usepackage{tikz}
\usepackage{xmpincl}
\usetikzlibrary{arrows}
\usepackage{pifont}

\usepackage{listings,color}

\definecolor{listcomment}{rgb}{0.0,0.5,0.0}
\definecolor{listkeyword}{rgb}{0.0,0.0,0.5}
\definecolor{listnumbers}{gray}{0.65}
\definecolor{listlightgray}{gray}{0.955}
\definecolor{listwhite}{gray}{1.0}


%% \setbeamertemplate{background canvas}{\includegraphics
%%    [width=\paperwidth,height=\paperheight]{./images/title.pdf}}

\AtBeginSection[]
{
\addtocounter{framenumber}{-1}
\begin{frame}
\frametitle{Sommaire}
\tableofcontents[currentsection]
\end{frame}}

%----------------------------------------------------------------------------------------
%	PAGE TITRE
%----------------------------------------------------------------------------------------
\title{Orfeo ToolBox users meeting and hackfest 2015}
\includexmp{images/cc}
\subtitle{Welcome, agenda, useful information}
\author{OTB development team}% date and event here
\date{3 - 5 june 2015, Toulouse}

\pgfdeclareimage[height=96mm,width=128mm]{background}{images/fondsClairSansLogo}
\pgfdeclareimage[height=0.2cm]{cc}{images/CC-licence.png}
\setbeamertemplate{background}{\pgfuseimage{background}}
\pgfdeclareimage[height=0.6cm]{logoIncrust}{images/logoIncrust}
\logo{
\begin{tabular}{p{0.22\textwidth}p{0.58\textwidth}p{0.1\textwidth}p{0.1\textwidth}}
\href{http://creativecommons.org/licenses/by-sa/3.0/}{\pgfuseimage{cc}}
& \vspace{-0.03\textwidth} \scriptsize{} % date and event here
&  & \href{http://www.orfeo-toolbox.org}{\pgfuseimage{logoIncrust}}\\
\end{tabular}
}

\begin{document}
\begin{frame}
\titlepage
\end{frame}

\section{Introduction}
\begin{frame}
\frametitle{Welcome!}
\begin{block}{Goal}
\begin{itemize}
\item Explore the way that OTB read/write geometry information
\item Explore the way that OTB stream the data
\item See how you can modify otb behaviour using extended filename features
\end{itemize}
\end{block}

\begin{block}{ToDo}
\begin{itemize}
\item Goal : Orthorectification or NDVI computation  of a Pleiades image
\item Easy?
\item In a efficient way...
\end{itemize}
\end{block}

\end{frame}

\section{Profiling with extractROI}

\begin{frame}[fragile]
\frametitle{ExtractROI basic}

\begin{block}{Command}
\begin{lstlisting}[language=bash]
#!/bin/bash
time otbcli_ExtractROI -in
IMG_PHR1A_MS_201202250025599_SEN_PRG_FC_5847-002_R1C1.TIF 
-out extract.tif 
-startx 1000 -starty 1000 
-sizex 5000  -sizey 5000 
\end{lstlisting}
\end{block}

\begin{block}{Profiling}
real	0m7.851s
\end{block}

\end{frame}

\begin{frame}
\frametitle{Pixel encoding (storage)}
\begin{itemize}
\item Input image is encoded on 16 bits
\item  By default output is written in float in OTB-Applications
\item  To write in unsigned int : add ``uint16'' after output image
\item Save Space and time (faster to write integers)
\item Can save more space using jpeg in tiff
\end{itemize}
\end{frame}


\begin{frame}[fragile]
\frametitle{ExtractROI (change output pixel type)}

\begin{block}{Command}
\begin{lstlisting}[language=bash]
#!/bin/bash
otbcli_ExtractROI -in  IMG_PHR1A_MS_201202250025599_SEN_PRG_FC_5847-002_R1C1.TIF 
-out extract.tif uint16 
-startx 1000 -starty 1000 -sizex 5000 -sizey 5000
\end{lstlisting}
\end{block}

\begin{block}{Profiling}
- real	0m5.010s
\end{block}

\end{frame}

\begin{frame}[fragile]
\frametitle{Dynamic range $\neq$ Storage}
\begin{itemize}
\item Dynamic range 12 bits and storage 16 bits for most HR sensors (like Pléiades)
\item Tif format support 12 bits/pixel
\item Possible with gdal options to write 12 bits tif
\item How to do this in OTB? Using extended filename mechanism
\item The reader and writer extended file name support is based on the same syntax, only the options are different.
\item Syntax: \begin{verbatim} Path/Image.ext?&key1=<value1>&key2=<value2> \end{verbatim}
\item IMPORTANT: Note that you'll probably need to "quote" the filename. 
\end{itemize}
\end{frame}

\begin{frame}[fragile]
\frametitle{ExtractROI with 12 bit per pixel}

\begin{block}{Command}
\begin{lstlisting}[language=bash]
#!/bin/bash
otbcli_ExtractROI -in  IMG_PHR1A_MS_201202250025599_SEN_PRG_FC_5847-002_R1C1.TIF 
-out "extract.tif?&gdal:co:NBITS=12" uint16 
-startx 1000 -starty 1000 -sizex 5000 -sizey 5000
\end{lstlisting}
\end{block}

\begin{block}{Profiling}
real	0m26.768s
Why?
\end{block}

\end{frame}

\begin{frame}[fragile]
\frametitle{Write large file 1/2}
\begin{itemize}
\item Ram parameter in OTB Applications: Available memory for processing (in MB)
\item Gdal got some caching capabilities. Writing large image file with OTB, it is
\item a good idea to extend size of this cache: export $GDAL_CACHEMAX=256$
\item Combination of streaming mode and internal file encoding  mode (tile/strip)
\end{itemize}
\end{frame}

\begin{frame}[fragile]
\frametitle{Write large file 2/2}
\begin{itemize}
\item see todo add link
\item You can manage the way that OTB streamed the data:
\item \begin{verbatim} "ortho.tif?&gdal:co:NBITS=12&streaming:type=tiled" \end{verbatim}
\item You can also manage the size of streaming pieces computation
\item \begin{verbatim}ortho.tif?&gdal:co:NBITS=12&streaming:type=tiled&streaming:sizevalue=2048" \end{verbatim} 
\item Never do tiled streaming and write stripped GeoTiff (default mode in gdal!)
\item \begin{verbatim} "ortho.tif?&gdal:co:NBITS=12&streaming:type=tiled
                       &streaming:sizevalue=2048&gdal:co:TILED=yes" \end{verbatim}
\end{itemize}
\end{frame}

\begin{frame}[fragile]
\frametitle{ExtractROI -> tiled tiff as output}

\begin{block}{Command}
\begin{lstlisting}[language=bash]
#!/bin/bash
 otbcli_ExtractROI -in
 IMG_PHR1A_MS_201202250025599_SEN_PRG_FC_5847-002_R1C1.TIF 
-out "extract.tif?&gdal:co:TILED=yes" uint16 
-startx 1000 -starty 1000 -sizex 5000 -sizey 5000
\end{lstlisting}
\end{block}

\begin{block}{Profiling}
real	0m2.997s
\end{block}

\end{frame}

\section{Bonus}
\begin{frame}[fragile]
\frametitle{Write a subset of the output}
\begin{itemize}
\item Method 1 : write the entire image and ExtractROI (facepalm)
\item Method 2 : use the extendedfilename box
\item \begin{verbatim} &box=<startx>:<starty>:<sizex>:<sizey> \end{verbatim}
\item \begin{verbatim} "ortho.tif?&gdal:co:NBITS=12&gdal:co:TILED=yes
                       &streaming:type=tiled&box=1000:1000:512:512" \end{verbatim}
\end{itemize}
\end{frame}

\end{document}
