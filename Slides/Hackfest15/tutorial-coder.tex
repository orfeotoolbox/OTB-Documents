%----------------------------------------------------------------------------------------
%	PACKAGES & THEMES
%----------------------------------------------------------------------------------------

\documentclass[8pt]{beamer}

\usepackage{etex}
\mode<presentation> {

\usetheme{Vilanova}
}



\usepackage[english]{babel}
\usepackage[utf8]{inputenc}
\usepackage{array}
\usepackage{chronology}
\let\CHRONOLOGY\chronology
\let\endCHRONOLOGY\endchronology
\def\chronology{\shorthandoff{;}\CHRONOLOGY}
\def\endchronology{\endCHRONOLOGY\shorthandon{;}}
\usepackage{pstricks}
\usepackage{graphicx}
\usepackage{booktabs}
\usepackage{amsmath,amssymb,amsthm}
\usepackage{xcolor}
\usepackage{textpos}
\usepackage{tikz}
\usepackage{xmpincl}
\usetikzlibrary{arrows}
\usepackage{pifont}

\usepackage{listings,color}

\definecolor{listcomment}{rgb}{0.0,0.5,0.0}
\definecolor{listkeyword}{rgb}{0.0,0.0,0.5}
\definecolor{listnumbers}{gray}{0.65}
\definecolor{listlightgray}{gray}{0.955}
\definecolor{listwhite}{gray}{1.0}


%% \setbeamertemplate{background canvas}{\includegraphics
%%    [width=\paperwidth,height=\paperheight]{./images/title.pdf}}

\AtBeginSection[]
{
\addtocounter{framenumber}{-1}
\begin{frame}
\frametitle{Sommaire}
\tableofcontents[currentsection]
\end{frame}}

%----------------------------------------------------------------------------------------
%	PAGE TITRE
%----------------------------------------------------------------------------------------
\title{Orfeo ToolBox users meeting and hackfest 2015}
\includexmp{images/cc}
\subtitle{Remote modules tutorial session}
\author{OTB development team}% date and event here
\date{3 - 5 june 2015, Toulouse}

\pgfdeclareimage[height=96mm,width=128mm]{background}{images/fondsClairSansLogo}
\pgfdeclareimage[height=0.2cm]{cc}{images/CC-licence.png}
\setbeamertemplate{background}{\pgfuseimage{background}}
\pgfdeclareimage[height=0.6cm]{logoIncrust}{images/logoIncrust}
\logo{
\begin{tabular}{p{0.22\textwidth}p{0.58\textwidth}p{0.1\textwidth}p{0.1\textwidth}}
\href{http://creativecommons.org/licenses/by-sa/3.0/}{\pgfuseimage{cc}}
& \vspace{-0.03\textwidth} \scriptsize{} % date and event here
&  & \href{http://www.orfeo-toolbox.org}{\pgfuseimage{logoIncrust}}\\
\end{tabular}
}

\begin{document}
\begin{frame}
\titlepage
\end{frame}

\begin{frame}
\frametitle{Organization of OTB sources}
\framesubtitle{Using OTB 5.0.0}

\begin{itemize}
\item For most OTB consumers/users, the modularization of the toolkit is relatively transparent. 
\item Presentation of the source organization (groups, modules)\ldots
\item CMake usage (How to configure OTB, options\ldots)
\item Modules enablement : dependency / default list / optional 3rd parties
\end{itemize}

\end{frame}

\begin{frame}
\frametitle{Extend OTB with remote module}
\framesubtitle{Setup steps}
\begin{itemize}
\item Clone module template (\url{https://github.com/orfeotoolbox/otbExternalModuleTemplate.git})
\item Module structure (CMake, sources, applications, tests)
\item Test the \texttt{otb\_fetch\_module()} CMake macro
\item Rename your external module (here : TutoHackfest)
  \begin{itemize}
  \item Update otb-module.cmake
  \item Update CMakeLists.txt : also change lib \texttt{name = module name}
  \end{itemize}
\item How to compile a module? \\ \texttt{make TutoHackfest}
\item How to compile tests and applications related to a module? \\ \texttt{make TutoHackfest-all}
\item How to launch tests related to a module? \\ \texttt{ctest -L TutoHackfest}
\end{itemize}

\end{frame}

\begin{frame}
\frametitle{Write your  application}
\framesubtitle{Tile fusion in OTB}
\begin{itemize}
\item Existing application has limitations regarding the number of tiles handled
\item \url{http://www.orfeo-toolbox.org/Applications/TileFusion.html}
\item Problem: number of reader instances
\item \url{https://bugs.orfeo-toolbox.org/view.php?id=1036}
\end{itemize}
\end{frame}

\begin{frame}
\frametitle{Write your  application}
\framesubtitle{Tutorial: new application for tile fusion}
\begin{itemize}
\item New tile fusion application with a different mechanism :
\begin{itemize}
\item Parse each tile separately
\item Write a VRT file that represent the mosaic
\item Instanciate a reader (using GDAL) and a writer to convert mosaic into plain image file
\end{itemize}
\item No more limitations regarding the number of reader instances
\item The VRT format from GDAL is powerful
\item Tutorial resources located in \texttt{OTB-Documents/Slides/Hackfest15/tutoCoder}
\item Example of a VRT file : \texttt{fusion.vrt}
\end{itemize}
\end{frame}

\begin{frame}
\frametitle{Write your  application}
\framesubtitle{Standard tile fusion application}
\begin{itemize}
\item Copy \texttt{tutoCoder/otbAdvancedTileFusion.cxx} to \texttt{OTB/Modules/Remote/yourRemoteModule/app}
\item Edit \texttt{app/CMakeLists.txt}
\item Edit \texttt{test/CMakeLists.txt} : change app name in \texttt{otb\_test\_application}
\item Try to compile \ldots
\item Fill the TODO's in the application code
\item Usefull links :
\begin{itemize}
\item \href{https://www.orfeo-toolbox.org/doxygen/classotb_1_1Wrapper_1_1Application.html}{otb::Wrapper::Application doxygen page}
\item \href{https://www.orfeo-toolbox.org/doxygen/classotb_1_1ImageFileReader.html}{otb::ImageFileReader doxygen page}
\item \href{http://www.gdal.org/gdal_vrttut.html}{VRT format tutorial}
\end{itemize}
\end{itemize}
\end{frame}

\begin{frame}
\frametitle{Write your  application}
\framesubtitle{Enhanced tile fusion application (gourou)}
\begin{itemize}
\item Mosaic images with a different dynamic (dataset \texttt{scaled\_R*C*.tif})
\item Use the otb::StreamingStatisticsVectorImageFilter to extract stats on each tile (you may need to add dependencies to your module)
\item In order to apply a rescaling on each tile, use \texttt{ComplexSource} type in the VRT
\item Use \texttt{ScaleOffset} and \texttt{ScaleRatio} for each tile 
\item Create a parameter in the application to enable/disable the dynamic adaptation
\item Usefull link : \href{https://www.orfeo-toolbox.org/doxygen/classotb_1_1StreamingStatisticsVectorImageFilter.html}{otb::StreamingStatisticsVectorImageFilter doxygen page}
\end{itemize}
\end{frame}


\end{document}
