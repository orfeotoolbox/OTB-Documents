\section*{Introduction}

\begin{frame}
\frametitle{Si vous ne retenez qu'une planche\ldots}
\begin{block}{L'Orfeo ToolBox est:}
\begin{itemize}
\item Une \textbf{bibliothèque de traitement d'images} pour la télédétection
\item \textbf{Un logiciel libre} diffusé sous licence Apache v2.0 (depuis OTB 6.0, précédement CeCILL-v2)
\item \textbf{Financée et développée par le CNES} (principalement)
\item Projet OSGeo depuis 2017
\item Écrite en \textbf{C++} sur la base d'\href{www.itk.org}{ITK} (imagerie médicale)
\item Construite sur les épaules de géants (ITK, GDAL, OSSIM, OpenCV\ldots)
\item Conçue pour traiter de \textbf{gros volumes de données} de manière transparente grâce au traitement par morceaux et à la parallélisation
\end{itemize}
\end{block}

\begin{center}
{\huge\textcolor{red}{\href{http://www.orfeo-toolbox.org}{orfeo-toolbox.org}}}
\end{center}

\end{frame}

\begin{frame}
\frametitle{Pourquoi un logiciel libre ?}

\begin{block}{Diffusion maximale}
L'OTB est un logiciel à destination de tous les utilisateurs d'imagerie
spatiale. Sa diffusion large contribue au rayonnement des missions (Pléiades, Sentinels\ldots)
\end{block}

\begin{block}{Qualité et efficacité}
Le domaine fonctionnel de l'OTB est vaste, son développement nécessite du temps
et de l'expertise. L'ouverture des sources favorise:
\begin{itemize}
\item L'appropriation et la validation par la communauté des utilisateurs,
\item Les contributions et les corrections de bugs par les utilisateurs,
\item La dissémination sur de multiples plateformes.
\end{itemize}
\end{block}

\begin{block}{Démarche scientifique}
L'OTB capitalise une partie de la R\&D du CNES en extraction d'information, l'ouverture des sources permet une démarche de \textbf{recherche reproductible}.
\end{block}

\end{frame}
